
\subsection{Необходимые стадии разработки}
\subsubsection{Техническое задание}
Этапы разработки:
\begin{my_enumerate}
	\item Обоснование необходимости разработки программы
		\begin{my_enumerate}
			\item постановка задачи
			\item сбор исходных материалов
			\item обоснование необходимости проведения научно-исследовательских работ
		\end{my_enumerate}
	\item Научно-исследовательские работы
		\begin{my_enumerate}
			\item определение структуры входных и выходных данных;
			\item предварительный выбор методов решения задач;
			\item определение требований к техническим средствам.
		\end{my_enumerate}
	\item Разработка и утверждение технического задания
		\begin{my_enumerate}
			\item определение требований к программе
			\item определение стадий, этапов и сроков разработки программы и документациик ней
		\end{my_enumerate}
\end{my_enumerate}


\subsubsection{Технический проект}
Этапы разработки:
\begin{my_enumerate}
	\item Разработка технического проекта
	\begin{my_enumerate}
		\item разработка технического проекта
		\item разработка структуры программы
	\end{my_enumerate}
	\item Утверждение технического проекта
	\begin{my_enumerate}
		\item разработка плана мероприятий по разработке программы
		\item разработка пояснительной записки
		\item согласование и утверждение технического проекта
	\end{my_enumerate}
\end{my_enumerate}


\subsubsection{Рабочий проект}
\begin{my_enumerate}
	\item Разработка программы
		\begin{my_enumerate}
			\item	программирование и отладка программы
			\item 	создание пакета инсталляции программы			
		\end{my_enumerate}
	\item Разработка программной документации
		\begin{my_enumerate}
			\item разработка программных документов в соответствии с требованиями ГОСТ 19.101-77.
		\end{my_enumerate}
     \item Испытания программы
   		\begin{my_enumerate}
	     	\item разработка, согласование и утверждение программы и методики испытаний
			\item корректировка программы и программной документации по результатамиспытаний
	     \end{my_enumerate}
\end{my_enumerate}

\subsection{Сроки работ и исполнители}

Приложение должно быть разработано к 1 мая 2017 года, студентом группы БПИ151 Куприяновым Кириллом.


% EOF
