Ниже приведены описания классов для Андроид клиента (Java)

\subsection{Классы и интерфейсы Java}

\subsubsection{LoginActivity}
Поля:\\
\begin{small}
    \begin{verbatim}
    private EditText etUsername;
    private EditText etPassword;
    private TextView tvRegister, tvHack;
    private Button btnLogin;
    private Button.OnClickListener btnLoginListener;
    private SharedPreferences sharedPrefs;
    private Map<String, String> userData;
    \end{verbatim}
\end{small}
Класс, отвечающий за отображение LoginActivity, проверку введённых данных и отправку запроса авторизации на сервер.

\subsubsection{MainActivity}
Поля:\\
\begin{small}
    \begin{verbatim}
    private boolean doubleBackToExitPressedOnce = false;
    private SwipeRefreshLayout swipeRefreshLayout;
    private static final String TAG_FRAGMENT_ONE = "fragment_one";
    private static final String TAG_FRAGMENT_TWO = "fragment_two";
    private Shop selectedShop = null;
    private View btnAdd;
    private View btnLoginLogout;
    public boolean homeActive = true;
    public int currentPage = 1;
    public String selectedCategory = "";
    public int totalItemsCount;
    public Parcelable itemsFragmentState;
    public Parcelable shopListsPreviewFragmentState;
    public ItemAdapter adapter;
    public ShopListsPreviewAdapter shopListsPreviewAdapter;
    public FetchData fetchData;
    public List<Shop> shops;
    public RequestQueue queue;
    private FragmentManager fragmentManager;
    public List<String> categories;
    private int currentFragmentId;
    \end{verbatim}
\end{small}
Основной класс программы, хранит состояния фрагментов, чтобы состояния прокрутки оставалась неизменной при переходе по фрагментам; запускает процессы отображения всех основных графических элементов, отвечает за логику перемещения внутри приложения и определяет поведение кнопки ``назад''.
\subsubsection{RegisterActivity}
Поля:\\
\begin{small}
    \begin{verbatim}
    private EditText etUsername, etPassword;
    private Button btnRegister;
    private Map<String, String> userData;
    \end{verbatim}
\end{small}
Класс, отвечающий за отображение RegisterActivity, проверку введённых данных и отправку запроса авторизации на сервер.

\subsubsection{ShopListActivity}
Поля:\\
\begin{small}
    \begin{verbatim}
    RecyclerView.LayoutManager layoutManager;
    RecyclerView rvShopList;
    SwipeRefreshLayout swipeRefreshLayout;
    public ItemAdapter adapter;
    public ShopList selectedShopList;
    public FetchData fetchData;
    public TextView tvTotalPrice;
    private View btnAdd;
    private TextView tvShopListName;
    \end{verbatim}
\end{small}
Класс, отвечающий за отображение ShopListActivity, загрузку списков покупок c серверa.


\subsubsection{ItemAdapter}
Поля:\\
\begin{small}
    \begin{verbatim}
    private List<Item> items;
    private List<Item> tmpItems;
    private ArrayList<Item> itemsCopy;
    private Context context;
    private Dialog itemFullPreview;
    private SparseBooleanArray expandState;
    private boolean type1Downloaded = false;
    private boolean type2Downloaded = false;
        \end{verbatim}
\end{small}
Класс, отвечающий за отображение каждого индивидуального Item'a.

\subsubsection{ShopListPreviewAdapter}
 Поля:\\
\begin{small}
    \begin{verbatim}
    public List<ShopList> shopLists;
    private Context context;
    \end{verbatim}
\end{small}
Класс, отвечающий за отображение каждого индивидуального ShopList'a.

\subsubsection{HomeFragment}
Поля:\\
\begin{small}
    \begin{verbatim}
    private RecyclerView rvItemList;
    private RecyclerView.LayoutManager layoutManager;
    \end{verbatim}
\end{small}
Класс, отвечающий за отрисовку главного фрагмента и toolbar'a.


\subsubsection{ShopListsPreviewFragment}
Поля:\\
\begin{small}
    \begin{verbatim}
    RecyclerView.LayoutManager layoutManager;
    RecyclerView rvShopLists;
    \end{verbatim}
\end{small}
Отображает ``превью'' всех пользовательских предметов и товаров из магазина в
виде карточек


\subsubsection{InternetUtil}
Методы:\\

\begin{small}
    \begin{verbatim}
    public static boolean isConnectedToInternet(Context context)
    \end{verbatim}
\end{small}
Класс, содержащий статический метод для проверки работы интернет-сети.

\subsubsection{JSONUtil}
Методы:\\

\begin{small}
    \begin{verbatim}
    public static JSONObject formPayload(Map<String, String> kvs)
    \end{verbatim}
\end{small}
Класс, формирующий хэдер JSON запроса.

\subsubsection{APIRequests}
Методы:\\

\begin{small}
    \begin{verbatim}
    public static String getStringPref(Context c, String key)
    public static void clearPrefs(Context c, String key)
    \end{verbatim}
\end{small}
Имеет статическую фабрику для получения Request Handler'a. Далее все действия
проводит RH. он добавляет к запросу URL, ReqponceListener и ErrorListener

\subsubsection{Config}
Поля:\\
\begin{small}
    \begin{verbatim}
    public static final String URL_CORE = "http://gcsales.ru/";
    public static final String URL_LOGIN = URL_CORE + "auth/login/";
    public static final String URL_REGISTER = URL_CORE + "auth/register/";
    public static final String URL_SALES_SHOP = URL_CORE + "api/shops/";
    public static final String URL_SHOPLIST = URL_CORE + "api/shoplist/";
    public static final String URL_SHOPLISTS_PREVIEW = URL_CORE + "api/shoplist?mode=preview";
    public static final String URL_SHOPLISTS = URL_CORE + "api/shoplist?mode=full";
    public static final String URL_ITEMS_ON_PAGE = "&page=";
    public static final String URL_ITEMS_IN_CATEGRY = "?category=";
    public static final String URL_CATEGORIES = "categories";
    public static final String URL_SL_ADD_ITEM = "additem?id=";
    public static final String URL_SL_ADD_CUSTOM_ITEM = "additem?custom=";
    public static final String URL_SL_DELETE_ITEM = "deleteitem?id=";
    public static final String KEY_USERNAME = "username";
    public static final String KEY_PASSWORD = "password";
    public static final String TAG_VOLLEY_ERROR = "VOLLEY";
    public static final String REQUESTS_CONTENT_TYPE = "application/json; charset=utf-8";
    public static final String BAD_API_AUTH_RESPONSE = "Wrong username/password.";
    public static final String SH_PREFS_NAME = "easy_sales_token";
    public static final String KEY_TOKEN = "user_token";
    public static final String DEF_NO_TOKEN = "NULL";
    public static final String KEY_ITEMS = "all_items";
    public static final String KEY_SHOPLIST_ITEMS = "shoplist_items";
    public static final String KEY_CURRENT_SHOPLIST = "com.hes.easysales.easysales.ShopList.";
    public static final int MAX_LEN_PREVIEW = 20;
    public static final int MAX_ITEMS_PREVIEW = 6;
\end{verbatim}
\end{small}
Класс с общими для всех даными.

\subsubsection{EndlessRCVScrollListener}
Поля:\\
\begin{small}
    \begin{verbatim}
    private LinearLayoutManager linearLayoutManager;
    private WeakReference<Activity> actvityRef;
\end{verbatim}
\end{small}
Класс, ``слушающий'' когда элемент RecyclerView дойдёт долистается конца. И подгружает ещё больше товаров.

\subsubsection{FetchData}
Поля:\\
\begin{small}
    \begin{verbatim}
    private ProgressBar pbLoading;

    // To prevent the leak of Context.
    //
    private WeakReference<Activity> activityRef;
    private WeakReference<SwipeRefreshLayout> swipeRefreshLayoutRef;
\end{verbatim}
\end{small}

Класс, который по мере возможности загружает всю необходимтую информацию.o

\subsubsection{Item}
Поля:\\
\begin{small}
    \begin{verbatim}
    private long id;
    private String name;
    private String category;
    private String imageUrl;
    private double oldPrice;
    private double newPrice;
    private String discount;
    private String dateIn;
    private String dateOut;
    private String condition;
    private Shop shop;
    // These fields are initialized explicitly using setters!
    // Factory method keeps them default.
    // However, Parcel keeps them for future reuse.
    //
    private boolean expandable = false;
    private boolean matched = false;
    private List<Item> matchingItems = new ArrayList<>();
\end{verbatim}
\end{small}

Класс, реализуюций POJO item Item.


\subsubsection{Shop}
Поля:\\
\begin{small}
    \begin{verbatim}
    private int id;
    private String alias;
    private String name;

    private List<String> categories;
\end{verbatim}
\end{small}

Класс, реализуюций POJO item Shop.


\subsubsection{ShopList}
Поля:\\
\begin{small}
    \begin{verbatim}
    private long id;
    private String name;
    private List<Item> items;
    private List<Item> customItems;
\end{verbatim}
\end{small}

Класс, реализуюций POJO item ShopList.


\subsubsection{ItemClickListener}
Метод:\\
\begin{small}
    \begin{verbatim}
    public void onClick(View v, int postition, boolean isLongClick);
\end{verbatim}
\end{small}
Интерфейс, определяющий поведение Item'ов при нажатии на них.



\subsection{Описание модулей Python}

\subsubsection{dixy\_crawler}

\begin{small}
    \begin{verbatim}
        class DixySpider(scrapy.Spider)
    \end{verbatim}
\end{small}

Класс, описывающий работу spider'a для магазина Дикси. Метод parse собирает с
веб-страницы (при помощи xpath селекторов) информацию, очищает её, передавая
модулю text\_processor, формирует DixyItems и отправляет их на вход модулю
pipelines.


\subsubsection{perekrestok\_crawler}

\begin{small}
    \begin{verbatim}
        class PerekrestokSpider(scrapy.Spider)
    \end{verbatim}
\end{small}

Класс, описывающий работу spider'a для магазина Перекресток. Метод parse собирает с
веб-страницы (при помощи xpath селекторов) информацию, очищает её, передавая
модулю text\_processor, формирует DixyItems и отправляет их на вход модулю
pipelines.

\subsubsection{text\_processor}

Методы:\\
\begin{small}
    \begin{verbatim}
remove_junk(s)
try_float(val)
concat(left, right, sep)
process(val)
process_str(val)
split_by(val, delim)
make_date(raw)
parse_date_in(s)
parse_date_out(s)
\end{verbatim}
\end{small}

\subsubsection{dixy\_selectors}

Поля:\\
\begin{small}
    \begin{verbatim}
    URLS = ['https://dixy.ru/akcii/skidki-nedeli']
    URL_CORE = 'https://dixy.ru'
    ITEM = '//div[contains(@class, "elem-product ")]'
    DESCRIPTION = 'div[@class="elem-product__description"]'
    INFO = 'div[@class="elem-product__info"]'
    PRICE_CONTAINER = INFO + '/div[@class="elem-product__price-container"]'
    PRICES = PRICE_CONTAINER + '/div[@class="elem-product__prices"]'
    IMG = INFO + '/div[@class="elem-product__image"]/img/@src'
    NAME = DESCRIPTION + '/div[contains(@class, "product-name")]/text()'
    CATEGORY = DESCRIPTION + '/div[@class="product-category"]/child::text()[2]'
    NEW_PRICE_LEFT = PRICES + '/div[@class="price-left"]/span/text()'
    NEW_PRICE_RIGHT = PRICES + '/div[@class="price-right"]/span/text()'
    OLD_PRICE_LEFT = PRICES + '/div[@class="price-right"]/div/span[@class="price-f\
    ull__integer"]/text()'
    OLD_PRICE_RIGHT = PRICES + '/div[@class="price-right"]/div/span[@class="price-\
    full__float"]/text()'
    DISCOUNT = PRICE_CONTAINER + '/div[contains(@class,"discount")]/span[@class="v\
    alue"]/text()' + ' | ' + PRICE_CONTAINER + '/div[@class="just-now"]/text()'
    CONDITION = PRICE_CONTAINER + '/div[contains(@class,"promopack")]/div[@class="\
    text"]/text()'
    DATE = 'div[contains(@class, "elem-badge-cornered")]/text()'
    NEXT_PAGE = '//div[contains(@class, "elem-pagination")]/a[contains(@class, "el\
    em-pagination__btn--next")]/@href'
    \end{verbatim}
\end{small}


\subsubsection{perekrestok\_selectors}

Поля:\\
\begin{small}
    \begin{verbatim}
    URLS = ['https://www.perekrestok.ru/catalog']
    URL_CORE = 'https://perekrestok.ru'
    CATEGORIES = '//a[@class="xf-catalog-categories__link"]/@href'
    CATEGORIES_TEXT = '//span[@class="xf-catalog-categories__text"]/text()'
    POST_CATEGORY = '//span[@class="xf-breadcrumbs__current"]/text()'
    CURRENT_CATEGORY = '//h1[@class="xf-caption__title"]/text()'
    ROOT_NODE = '//ul[@id="catalogItems"]'
    ITEM = 'li/div[contains(@class, "xf-product")]'
    NAME = 'div[@class="xf-product__title xf-product-title"]/a[@class="xf-product-\
    title__link js-product__title"]/text()'
    IMG = 'figure/a/img/@data-src'
    NEW_PRICE = 'div/div[contains(@class, "xf-product-cost__current")]/@data-cost'
    OLD_PRICE = 'div/div[contains(@class, "xf-price xf-product-cost__prev")]/@data-cost'
    NEXT_PAGE = '//a[contains(@class, "xf-paginator__item js-paginator__next")]/@href'
    \end{verbatim}
\end{small}


\subsubsection{pipelines}

Классы:\\
\begin{small}
    \begin{verbatim}
    class GeneralProcessPipeline(object)
    class JsonWriterPipeline(object)
    class DataBaseWriterPipeLine(object)
    \end{verbatim}
\end{small}

\subsubsection{config}

Поля:\\

\begin{small}
    \begin{verbatim}
    const =\
    {
    'SHOPS_INFO_API': 'http://46.17.44.125/api/shops',
    'ADD_ITEM_API': 'http://localhost/api/shops',
    'URI_JSON_OUT': 'out/items.json',
    'REQUEST_HEADERS': {'Content-Type': 'application/json'},
}
    \end{verbatim}
\end{small}
