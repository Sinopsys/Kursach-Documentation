% При программировании на VueJS, каждый модуль (компонента) предсавляет собой
% отдельный файл, в котором инкапсулированы 

\subsection{Описание компонент веб-приложения}

\subsubsection{Discounts.vue}
Поля:
\begin{minted}{javascript}
  props: [ 'shopId', 'currentCategory', 'currentPage' ],
  data: function() {
    return {
      categories: [],
      items: [],
      shoplists: [],
      numPages: 0,
      user: auth.user
    }
  },
\end{minted}
Компонента для отображения списка текущих акций для разных магазинов.

\subsubsection{Header.vue}
Поля:
\begin{minted}{javascript}
  data: function() {
    return {
      shops: [],
      user: auth.user
    }
  }
\end{minted}
Компонента для отображения меню навигации вверху страницы.

\subsubsection{Item.vue}
Поля:
\begin{minted}{javascript}
  props: ['item', 'shoplists'],
  data: function() {
    return {
      maxNameLength: 80,
      showImage: true,
      user: auth.user
    };
  },
\end{minted}
Компонента для отображения товара магазина в списке акций.

\subsubsection{ItemSmall.vue}
Поля:
\begin{minted}{javascript}
  props: ['item', 'btnText', 'index'],
  data: function() {
    return {
      maxNameLength: 40
    }
  }, 
\end{minted}
Компонента для отображения товара магазина в списке покупок.

\subsubsection{Login.vue}
Поля:
\begin{minted}{javascript}
  data: function() {
    return {
      username: '',
      password: '',
      error: ''
    }
  },
\end{minted}
Компонента для отображения формы входа в систему.

\subsubsection{Register.vue}
Поля:
\begin{minted}{javascript}
  data: function() {
    return {
      username: '',
      password: '',
      error: ''
    }
  },
  },
\end{minted}
Компонента для отображения формы регистрации в системе.

\subsubsection{ShopList.vue}
Поля:
\begin{minted}{javascript}
  data: function() {
    return {
      shoplist: {},
      customItem: ''
    }
  },
\end{minted}
Компонента для отображения списка покупок.

\subsubsection{ShopListPreview.vue}
Поля:
\begin{minted}{javascript}
  data: function() {
    return {
      shoplists: [],
      shoplist: '',
      user: auth.user
    }
  },
\end{minted}
Компонента для отображения контейнера для карточек-превью списков покупок.

\subsubsection{ShopListCard.vue}
Поля:
\begin{minted}{javascript}
  props: ['shoplist'],
  data: function() {
    return {
      headerVariant: {
        bg: 'primary',
        fg: 'white'
      },
      maxNameLength: 40,
      maxItemsCount: 3
    }
  },
\end{minted}
Компонента для отображения карточки-превью списка покупок.

\subsubsection{ShopListCustomItem.vue}
Поля:
\begin{minted}{javascript}
  props: ['index', 'item'],
\end{minted}
Компонента для отображения пользовательской позиции в списке покупок.

\subsection{Описание модулей серверной части приложения}

\subsubsection{models/account.js}
Объектная модель для таблицы account в базе данных.

\subsubsection{models/custom\_item.js}
Объектная модель для таблицы custom\_item в базе данных.

\subsubsection{models/item.js}
Объектная модель для таблицы item в базе данных.

\subsubsection{models/shop.js}
Объектная модель для таблицы shop в базе данных.

\subsubsection{models/shoplist\_item.js}
Объектная модель для таблицы shoplist\_item в базе данных.

\subsubsection{models/shoplist.js}
Объектная модель для таблицы shoplist в базе данных.

\subsubsection{models/index.js}
Модуль для указания ассоциаций между моделями, описанными выше.

\subsubsection{routes/auth.js}
Модуль для обработки REST запросов для логина и регистрации.

\subsubsection{routes/api/sales.js}
Модуль для обработки REST запросов для работы со списком акций.

\subsubsection{routes/api/shoplist.js}
Модуль для обработки REST запросов для работы со списком покупок.
