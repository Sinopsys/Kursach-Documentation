При программировании на Unity, каждый скрипт предсавляет собой отдельный класс, в котором, как правило, есть 2 метода: Start и Update. Start вызывается 1 раз при запуске скрипта, а Update вызывается 1 раз за фрейм. 

\subsection{WalkByLook}
Поля:\\
\begin{small}
    \begin{verbatim}
    public float angle1 = 30F, angle2 = 80F;
    public float speed = 5F;
    private bool moveForward;
    private CharacterController cc;
    public Transform vrCam;
    private bool lookingToObject = false;
    private bool mustLookToTheObject = false;
    public AudioSource[] footSteps;
    private bool shouldMove = false, playing = false;
    \end{verbatim}
\end{small}
Класс, созданный для передвижения игрока при помощи угла наклона головы по оси Х.

\subsection{BellBehaviour}
Поля:\\
\begin{small}
    \begin{verbatim}
public Material bellHoverMaterial;
public Material defaultBellMaterial;
private AudioSource audioSrc;
    \end{verbatim}
\end{small}
Отвечает за поведение колокольчика при наведении на него reticle и нажатии (загрузки нижнего индикатора).
\subsection{BellPuzzleLauncher}
Поля:\\
\begin{small}
    \begin{verbatim}
public Transform handMountingPosition;
private bool hasBell = false;
private float angle;
public Transform vrCam;
public GameObject bellPuzzle;
public Transform firstToComplete, secondToComplete;
private bool firstPut = false, secondPut = false;
    \end{verbatim}
\end{small}
Прикрепляется к невидимому кубу с коллайдером, который при столкновении с персонажем и наведении им указателя на место головоломки вешает колокольчик из руки, и, если все 5 на месте, запускает головоломку. 

\subsection{KeyDoorOpener}
Поля:\\
\begin{small}
    \begin{verbatim}
public GameObject door, final;
private static bool opened = false;
private float timeToOpen = 1F;
public Transform handMountingPosition;
    \end{verbatim}
\end{small}

Класс, отвечающий за открытие двери когда в руке у игрока находится ключ.

\subsection{LastPuzzleLogic}
Поля:\\
\begin{small}
    \begin{verbatim}
    char[] numbers;
    bool solved;
    float step;
    string key = "4118956";
    public GameObject Key;
        \end{verbatim}
\end{small}

Класс, отвечающий за поведение последней головоломки с ключем на книжном шкафу.

\subsection{MainMenu}
 Поля:\\
\begin{small}
    \begin{verbatim}
        public GameObject character, story;
   
    \end{verbatim}
\end{small}

Класс главного меню, прикрепляется к двум лэйблам <<PLAY>> и <<EXIT>>.

\subsection{ModChanger}
Поля:\\
\begin{small}
    \begin{verbatim}
    public GameObject modeChooser;
    public static bool vrModeEnabled = false;
    private bool checked_ = false;ы
    \end{verbatim}
\end{small}
Класс, отвечающий за выбор режима (VR Mode или Normal Mode)


\subsection{NumberChanger}
Поля:\\
\begin{small}
    \begin{verbatim}
    int n;
    public float timeDelay = 1F;
    float currentTime = 0F;
    bool gazedAt = false;
    \end{verbatim}
\end{small}
Класс, отвечающий за изменение каждой цифры в последней головоломке.  


\subsection{OpenDoorAndLoadScene}
Поля:\\
\begin{small}
    \begin{verbatim}
    public GameObject door;
    public static bool opened = false;
    public float timeToOpen = 0.6F;
    \end{verbatim}
\end{small}
Класс, отвечающий за открытие двери с TransitScene для входа в дом.

\subsection{PickUpObject}
Поля:\\
\begin{small}
    \begin{verbatim}
    public Vector3 handPosition;
    public Vector3 handRotation;
    Vector3 oldScale;
    public float angle1 = 275F, angle2 = 303F;
    public Transform handMountingPosition;
    public Transform vrCam;
    private bool tilted;
    bool pickedUp = false;
    \end{verbatim}
\end{small}
Класс, отвечающий за перемещение объекта с места, где он лежал, в руку игроку.

\subsection{StoryLogic}
Поля:\\
\begin{small}
    \begin{verbatim}
    public GameObject character, vrCam, guide, tryIt, glhf, stone, constraint;
    private bool freezeMove = true, thrown = false, shown = false;
    \end{verbatim}
\end{small}
Класс, отвечающий за последовательность показа приветственных текстов и обучающего фрагмента вначале.

\subsection{Unscrew}
Поля:\\
\begin{small}
    \begin{verbatim}
    bool unscrewed = false, hasScrewDriver = false;
    public Transform handMountingPosition;
    \end{verbatim}
\end{small}
Класс, отвечающий за откручивание колокольчика со стены при помощи отвертки в руке игрока.

\subsection{VRSlider}
Поля:\\
\begin{small}
    \begin{verbatim}
    public float fillTime = 2f;
    public int Scene = 1;
    private Slider mySlider;
    private float timer;
    private bool gazedAt;
    private Coroutine fillBarRoutine;
    \end{verbatim}
\end{small}
Класс, отвечающий за анимацию слайдера - индикатора внизу экрана.

