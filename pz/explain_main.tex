\documentclass[
%a4paper,12pt
encoding=utf8
]{twoeskd}

% Packages required by doxygen
\usepackage{fixltx2e}
\usepackage{calc}
\usepackage{doxygen}
\usepackage[export]{adjustbox} % also loads graphicx
\usepackage{graphicx}
\usepackage[utf8]{inputenc}
\usepackage{makeidx}
\usepackage{multicol}
\usepackage{multirow}
\PassOptionsToPackage{warn}{textcomp}
\usepackage{textcomp}
\usepackage[nointegrals]{wasysym}




% NLS support packages
\usepackage[T2A]{fontenc}
\usepackage[russian]{babel}

% Font selection
\usepackage{courier}
\usepackage{amssymb}
\usepackage{sectsty}
\renewcommand{\familydefault}{\sfdefault}
\newcommand{\+}{\discretionary{\mbox{\scriptsize$\hookleftarrow$}}{}{}}

% Page & text layout
\usepackage{geometry}
\tolerance=750
\hfuzz=15pt
\hbadness=750
\setlength{\emergencystretch}{15pt}
\setlength{\parindent}{0cm}
\setlength{\parskip}{0.2cm}
\makeatletter
\makeatother

% Headers & footers
% \usepackage{fancyhdr}
% \renewcommand{\sectionmark}[1]{%
%   \markright{\thesection\ #1}%
% }

% Indices & bibliography
\usepackage{natbib}
\usepackage[titles]{tocloft}
\setcounter{tocdepth}{3}
\setcounter{secnumdepth}{5}
\makeindex

\newcommand\tab[1][1cm]{\hspace*{#1}}
\usepackage{hyperref}
\hypersetup{
	colorlinks,
	citecolor=black,
	filecolor=black,
	linkcolor=black,
	urlcolor=black
}
% Custom commands
\newcommand{\clearemptydoublepage}{%
  \newpage{\pagestyle{empty}\cleardoublepage}%
}
\renewcommand{\DoxyLabelFont}{%
  \fontseries{bc}\selectfont%
}
\newcommand\degr{$^\circ$}

% Custom packages
\usepackage{pdfpages}


\setlength{\parindent}{0cm}
\setlength{\parskip}{0.2cm}

\usepackage{listings}

% debug to see the frame borders
% from https://en.wikibooks.org/wiki/LaTeX/Page_Layout
% \usepackage{showframe}

% change style of titles in \section{}
\usepackage{titlesec}
\titleformat{\section}[hang]{\huge\bfseries\center}{\thetitle.}{1em}{}
\titleformat{\subsection}[hang]{\Large\raggedright}{\thetitle.}{1em}{\underline}
\titleformat{\subsubsection}[hang]{\large\raggedright}{\thetitle.}{1pt}{}

% Packages for text layout in normal mode
% \usepackage[parfill]{parskip} % автоматом делает пустые линии между параграфами, там где они есть в тексте
% \usepackage{indentfirst} % indent even in first paragraph
\usepackage{setspace}	 % controls space between lines
\setstretch{1} % space between lines
\setlength\parindent{0.9cm} % size of indent for every paragraph
\usepackage{csquotes}% превратить " " в красивые двойные кавычки
\MakeOuterQuote{"}


% this makes items spacing single-spaced in enumerations.
\newenvironment{my_enumerate}{
\begin{enumerate}
  \setlength{\itemsep}{1pt}
  \setlength{\parskip}{0pt}
  \setlength{\parsep}{0pt}}{\end{enumerate}
}


% Custom commands
% configure eskd
\titleTop{
\textbf{\Large ПРАВИТЕЛЬСТВО РОССИЙСКОЙ ФЕДЕРАЦИИ \\
НАЦИОНАЛЬНЫЙ ИССЛЕДОВАТЕЛЬСКИЙ УНИВЕРСИТЕТ \\
«ВЫСШАЯ ШКОЛА ЭКОНОМИКИ» } \\
\vspace*{0.2cm}
{\small Факультет компьютерных наук \\
Департамент программнoй инженерии \\
}
}
\titleDesignedBy{Студент группы БПИ 151 НИУ ВШЭ}{Куприянов К.И.}
\titleAgreedBy{%
\parbox[t]{7cm} {
Доцент департамента \\
программной инженерии \\
факультета компьютерных наук \\
канд. техн. наук \\
}}{Гринкруг Е. М.}
\titleApprovedBy{
\parbox[t]{10cm} {
Академический руководитель \\
образовательной программы \\
«Программная инженерия» \\
профессор департамента программной \\
инженерии канд. техн. наук \\
}}{Шилов В. В.}
\titleName{Игра — Эскейп Квест с Использованием Очков Виртуальной Реальности}
\workTypeId{RU.17701729.509000 81 01-1}

\titleSubname{Пояснительная записка}


%===== C O N T E N T S =====
\begin{document}

% Titlepage & ToC
\pagenumbering{arabic}

\section{Аннотация}


В данном программном документе приведена пояснительная записка к программе 
"Клиент-Серверное Веб-Приложение для Управления Скидками в Розничных Сетях".
В данном программном документе, в разделе «Введение» указано наименование 
программы, краткое наименование программы и документы, на основании которых 
ведется разработка.
В разделе «Назначение и область применения» указано функциональное назначение 
программы и краткая характеристика области применения программы.
В данном программном документе, в разделе «Технические характеристики» 
содержатся следующие подразделы:
\begin{itemize}
    \item постановка задачи на разработку программы
    \item описание алгоритма и функционирования программы с обоснованием выбора 
    схемы алгоритма решения задачи и возможные взаимодействия программы с 
    другими программами
    \item описание и обоснование выбора состава технических и программных 
    средств
\end{itemize}
Так же к документу прикреплены приложения. 
Настоящий документ разработан в соответствии с требованиями:
\begin{my_enumerate}
    \item ГОСТ 19.101-77 Виды программ и программных документов [4];
    \item ГОСТ 19.102-77 Стадии разработки [1];
    \item ГОСТ 19.103-77 Обозначения программ и программных документов [4];
    \item ГОСТ 19.104-78 Основные надписи [6];
    \item ГОСТ 19.106-78 Требования к программным документам, выполненным 
    печатным 
    способом [7];
    \item ГОСТ 19.404-79 Пояснительная записка. Требования к содержанию и 
    оформлению [3].
\end{my_enumerate}
    Изменения к данному Техническому заданию оформляются согласно ГОСТ 
19.603-78 [8]


\newpage
\pagenumbering{arabic}
\tableofcontents
% \pagenumbering{arabic}

% --- add my custom headers ---
\newpage
\section{Введение}
\subsection{Наименование программы}
Наименование программы на русском: 
``Клиент-Серверное Android-Приложение для Управления Скидками в Розничных Сетях''. \\
Наименование на английском: 
``The Client-Server Android Application for Managing the Products' Discounts in Retail Networks''. \\


\subsection{Краткая характеристика} 
Программа предназначена для пользователей смартфонов на базе платформы
Android.  Цель работы - создать удобное приложения для отслеживания,
составления списка покупок акционных товаров, управления им, а так же
просмотра всех акций в нескольких магазинах. Это позволит пользователям
экономить свои средства на ежедневных покупках.  Серверная часть
приложения должна представлять из себя web-crawler для сбора данных с
сайтов розничных сетей. Должен быть реализован механизм взаимодействия с
клиентами посредством обменивания файлами в формате JSON.



\newpage
\section{Назначение разработки}
\subsection{Функциональное назначение}
К функциональным возможностям программы относятся: возможность выбора между VR Mode и Normal Mode, функционал для передвижения игрока по миру виртуальной реальности и взаимодействия с ней и предметами, находящимися на сцене (см. терминологию); переключение между сценами. Все взаимодействия с виртуальным миром и предметами в нем должны осуществляться без нажатия кнопки. Пояснение: у Google Cardboard'ов есть несколько разных моделей: у одних есть кнопка (триггер), а у других нет. Приложение спроектировано так, чтобы обеспечить максимальную совместимость со всеми девайсами. 

\subsection{Эскплутационное назначение}
Программа предназначена для запуска на мобильных устройствах операционной системы Андроид, поддерживающих технологию VR. Программа является досуговым приложением, позволяющим пользователю окунуться в мир виртуальной реальности и решать в нем головоломки, чтобы найти ключ, с помощью которого можно отпереть дверь и выбраться из заперти. 

\newpage
\section{Технические характеристики}
\subsection{Постановка задачи на разработку программы}
Программа должна соответствовать требованиям, представленным в 
Техническом Задании.

\bigskip
Задачи работы (серверная часть приложения):

\smallskip
\begin{my_enumerate}
  \item Реализовать REST API:
    \begin{my_enumerate}
      \item Обработка запросов для работы со списками покупок пользователей
      \item Обработка запросов для кроулера
      \item Обработка запросов для получения списка актуальных акционных
        товаров
      \item Обработка запросов для авторизации пользователей
    \end{my_enumerate}
  \item Спроектировать базу данных для хранения:
    \begin{my_enumerate}
      \item Акционных товаров
      \item Аккаунтов пользователей
      \item Списков покупок пользователей
    \end{my_enumerate}
\end{my_enumerate}

\bigskip
Задачи работы (клиентская часть приложения):

\smallskip
\subsubsection{Состав выполняемых функций. Клиентская часть.}
\begin{my_enumerate}
  \item Реализовать возможность просмотра списка доступных магазинов с акционными товарами
  \item Реализовать представление текущих акций для конкретного магазина:
    \begin{my_enumerate}
      \item В виде общего списка
      \item По категориям
    \end{my_enumerate}
  \item Реализовать возможность регистрации и авторизации пользователей в системе для
    работы со списком покупок
  \item Реализовать cоздание и удаление списка покупок
  \item Реализовать работу со списком покупок:
    \begin{my_enumerate}
      \item Добавление и удаление товаров магазинов
      \item Добавление и удаление пользовательских позиций
      \item Рекомендация товаров магазинов на основе пользовательских позиций 
    \end{my_enumerate}
\end{my_enumerate}

\subsection{Описание алгоритма и функционирования программы}

\subsubsection{Схема базы данных}
\begin{figure}[h]
    \centering
    \includegraphics[width=\textwidth]{./pics/database.png}
    \caption{\small{Схема базы данных}}
    \label{database}
\end{figure}

Пожалуй, главной сущностью в схеме является "item". Именно в эту таблицу
записываются товары, собранные кроулером, и из этой таблицы извлекаются данные
для отображения в клиентских приложениях. Каждый товар также связан с
магазином, которому он принадлежит. (Один магазин может иметь много товаров).

Как видно из Рис. 1, пользователь может иметь несколько списков покупок,
которые в свою очередь могут содержать много товаров.

\subsubsection{Список покупок}
Список покупок имеет следующую структуру:
\begin{my_enumerate}
  \item Имя списка покупок (Например, "Завтрак")
  \item Список товаров магазинов (Например, "Молочный коктейль Чудо детки
    шоколад; клубника 2,5\%, 200 мл")
  \item Пользовательские позиции (Например, "Сок")
  \item Итоговая сумма покупок
\end{my_enumerate}

При добавлении пользовательской позиции в список покупок, система предложит
пользователю актуальные товары из магазинов.\\
Так, например, для позиции "Сок" будут предложены следующие товары, 
которые пользователь может добавить в список товаров магазинов:
\begin{my_enumerate}
\item Сок Сады Придонья Яблоко зеленое 125мл
\item Сок Добрый Яблочный 2л
\item Десерт Джелео многослойный с соком вишня-персик-яблоко 0,4\%, 150 г
\end{my_enumerate}

Список товаров магазинов и предложенные товары на основе пользовательских
позиций для удобства группируются по магазинам.
Добавлять товары в список покупок могут только авторизованные пользователи.


\subsubsection{Описание API}
В этом разделе перечислены все endpoint'ы для реализованного API.
Для каждого endpoint'а приведено краткое описание и пример ответа в формате
json.

\textbf{Работа со списком акций}

\noindent
/api/shops -- получить список всех магазинов\\
GET-запрос\\
Пример ответа:
\begin{minted}{json}
[
    {
        "id": 1,
        "alias": "dixy",
        "name": "Дикси"
    },
    {
        "id": 2,
        "alias": "perekrestok",
        "name": "Перекресток"
    }
]
\end{minted}

\noindent
/api/shops -- загрузить новые товары на сервер. Этот endpoint нужен для
кроулера, чтобы записать собранные акции в базу данных. \\
POST-запрос\\
Тело запроса:
\begin{minted}{json}
{
    "name": "Компот Д из персиков, 580 мл",
    "category": "Консервы, соусы",
    "oldPrice": 137,
    "newPrice": 99.99,
    "dateIn": "2018-04-02",
    "dateOut": "2018-04-08",
    "crawlDate": "2018-04-02",
    "condition": "-",
    "image": null,
    "imageUrl": "https://dixy.ru/upload/iblock/fe2/2000183687.jpg",
    "discount": "-27",
}
\end{minted}

\noindent
/api/shops/:shopid -- получить список товаров для данного магазина\\
GET-запрос\\
Параметры URL:
\begin{my_enumerate}
  \item shopid -- id магазина
\end{my_enumerate}
Параметры запроса (опционально):
\begin{my_enumerate}
\item category -- название категории (по умолчанию возвращаются товары всех
  категорий)
  \item page -- номер страницы (по умолчанию 1)
\end{my_enumerate}
Пример ответа:\\
/api/shops/1?category=Кулинария,\%20заморозка,\%20мороженое\&page=1
\begin{minted}{json}
{
  "count": 192,
  "rows":
  [
      {
          "id": 565,
          "name": "Мороженое Жемчужина России эскимо миндаль-карамель, 80 г",
          "category": "Кулинария, заморозка, мороженое",
          "oldPrice": 52.9,
          "newPrice": 26.45,
          "dateIn": "2018-03-15",
          "dateOut": "2018-03-28",
          "crawlDate": "2018-03-17",
          "condition": "-",
          "image": null,
          "imageUrl": "https://dixy.ru/upload/iblock/814/2000148579.jpg",
          "discount": "1+1",
          "shopId": 1
      }
  ],
  "numPages": 1
}
\end{minted}
Здесь:\\
\begin{my_enumerate}
  \item count -- общее количество товаров для данного магазина\\
  \item numPages -- всего страниц с товарами. Расчитывается как:
    $$
    Math.ceil(count / ITEMS\_PER\_PAGE), ITEMS\_PER\_PAGE=30
    $$
  \item rows -- список с товарами\\
\end{my_enumerate}
Видно, что сервер отдает данные постранично. Т.е. вместо того, чтобы
отдать сразу все 192 товара, сервер возвращает страницы, содержащие по 30 товаров.
Поля dateIn и dateOut у объекта item нужны для того, чтобы загружать только
актуальные товары из базы. Т.е. выполняется запрос вида:
\begin{minted}{sql}
SELECT * FROM item where date_in <= currDate AND date_out >= currDate
\end{minted}

\noindent
/api/shops/:shopid/categories -- получить список категорий для данного магазина\\
GET-запрос\\
Параметры URL:
\begin{my_enumerate}
  \item shopid -- id магазина
\end{my_enumerate}
Пример ответа:
\begin{minted}{json}
["Фермерские продукты","Хлеб, сладости, снеки","Овощи, фрукты, грибы","Товары
для животных","Товары для мам и детей","Консервы, орехи, соусы","Макароны,
крупы, специи","Мясо, птица, деликатесы","Соки, воды, напитки","Молоко, сыр,
яйца","Авто, дом, сад, кухня","Наши марки","Рыба и морепродукты","Красота,
гигиена, бытовая химия","Кофе, чай, сахар","Замороженные продукты","Здоровый
выбор"]
\end{minted}

\textbf{Работа со списком покупок}

\noindent
Все запросы из этого раздела должны быть авторизваны. (см. раздел Авторизация)

\noindent
/api/shoplist -- получить списки покупок для данного пользователя\\
GET-запрос\\
Параметры запроса:
\begin{my_enumerate}
  \item mode -- режим для загрузки списков покупок
    \begin{my_enumerate}
      \item preview -- списки покупок загружаются в сжатом виде
      \item full -- списки покупок загружаются в полном виде
    \end{my_enumerate}
    Если параметр не задан, то загружаются списки покупок без товаров
\end{my_enumerate}
Примеры ответа:\\
/api/shoplist
\begin{minted}{json}
[
    {
        "id": 17,
        "name": "Купить"
    },
    {
        "id": 18,
        "name": "Чё купить"
    },
    {
        "id": 19,
        "name": "Zzz"
    }
]
\end{minted}
/api/shoplist?mode=preview
\begin{minted}{json}
[
    {
        "id": 17,
        "name": "Купить",
        "items": [
            "Мыло Солнышко хозяйственное с ароматом лимона 140г",
            "Крем-мыло жидкое Особая серия овсяное молочко, 1000 г"
        ],
        "customItems": [
            "Мыло"
        ]
    },
    {
        "id": 18,
        "name": "Чё купить",
        "items": [
            "Мешок для мусора Фрекен БОК 35л 15шт",
            "Сок Добрый Яблочный 2л",
            "Сосиски Молочные Велком, 530 г",
            "Нектар Добрый апельсиновый; мультифруктовый; персик-яблоко; груша; виноград-гранат; сок яблочный осветленный; томатный, 1 л"
        ],
        "customItems": [
            "сок",
            "колбаса"
        ]
    }
]
\end{minted}
/api/shoplist?mode=full
\begin{minted}{json}
[
    {
        "id": 17,
        "name": "Купить",
        "items": [
            {
                "id": 294,
                "name": "Мыло Солнышко хозяйственное с ароматом лимона 140г",
                "category": "Красота, гигиена, бытовая химия",
                "oldPrice": 0,
                "newPrice": 27.9,
                "dateIn": "2018-03-30",
                "dateOut": "2018-04-20",
                "crawlDate": "2018-03-30",
                "condition": null,
                "image": null,
                "imageUrl": "https://perekrestok.ru/src/product.file/list/image/21/67/16721.jpeg",
                "discount": null,
                "shop": {
                    "id": 2,
                    "alias": "perekrestok",
                    "name": "Перекресток"
                }
            },
            {
                "id": 73880,
                "name": "Крем-мыло жидкое Особая серия овсяное молочко, 1000 г",
                "category": "Непродовольственные товары",
                "oldPrice": 89.9,
                "newPrice": 44.95,
                "dateIn": "2018-04-02",
                "dateOut": "2018-04-08",
                "crawlDate": "2018-04-02",
                "condition": "-",
                "image": null,
                "imageUrl": "https://dixy.ru/upload/iblock/267/2000267119.jpg",
                "discount": "1+1",
                "shop": {
                    "id": 1,
                    "alias": "dixy",
                    "name": "Дикси"
                }
            }
        ],
        "customItems": [
            {
                "id": 24,
                "name": "Мыло",
                "shoplistId": 17,
                "matchingItems": [
                    {
                        "id": 187,
                        "name": "Крем-мыло жидкое Dove Красота и уход 250мл",
                        "category": "Красота, гигиена, бытовая химия",
                        "oldPrice": 0,
                        "newPrice": 259,
                        "dateIn": "2018-03-30",
                        "dateOut": "2018-04-20",
                        "crawlDate": "2018-03-30",
                        "condition": null,
                        "image": null,
                        "imageUrl": "https://perekrestok.ru/src/product.file/list/image/74/74/7474.jpeg",
                        "discount": null,
                        "shop": {
                            "id": 2,
                            "alias": "perekrestok",
                            "name": "Перекресток"
                        }
                    }
                  ]
            }
        ]
    }
]
\end{minted}
В этом примере видно, что помимо списка items, соответствующего товарам
магазинов, также есть список customItems, которой представляет пользовательские
позиции. У каждой пользовательской позиции есть список matchingItems, который
содержит товары магазинов, удовлетворяющих данной позиции. Также у каждого
товара есть поле shop, с помощью которого можно определить, какому магазину
принадлежит данный товар, чтобы, например, сгруппировать товары в списке покупок
по магазинам.

\subsubsection{Авторизация}
Для доступа к списку покупок пользователю необходимо авторизоваться.
Авторизация на сервере выполняется следующим образом:
\begin{my_enumerate}
  \item Пользователь авторизуется в системе, вводя свои логин и пароль
  \item Если данные верны, система возвращает токен
  \item Полученный токен клиентское приложение добавляет в заголовок
    каждого защищенного запроса
\end{my_enumerate}

Токен имеет формат JWT и представлен в следующем виде:
\begin{center}
  xxxxx.yyyyy.zzzzz
\end{center}
где\\
xxxxx -- Header (заголовок). В нем содержится метаинформация, такая как алгоритм шифрования и тип токена (как правило это JWT)\\
yyyyy -- Payload -- здесь содержится основная информация, в частности время создания токена, имя и id пользователя, получившего токен.\\
zzzzz -- Signature -- сигнатура токена. Позволяет проверить, что токен не был
подменен и является валидным. Получается в результате применения SHA256 функции:
SHA256(Header, Payload, Secret key), где Secret key -- секретный ключ, хранящийся на сервере. \\
Каждая часть токена закодирована в формат Base64.

Для авторизации в API есть два endpoint'a
\begin{my_enumerate}
  \item /auth/register -- для регистрации в системе
  \item /auth/login -- для входа в систему
\end{my_enumerate}

В обоих случаях в теле реквеста передается следующая информация: 
\begin{minted}{json}
  {
    "username": "Имя пользователя",
    "password": "Пароль"
  }
\end{minted}

В случае регистрации возвращается статус 200, если регистрация прошла успешно
или статус 500, если пользователь ввел короткое имя или пароль, либо такое
имя пользователя уже существует.

В случае успешного логина возвращается токен. Клиентские приложения должны
сохранить его, чтобы в следующий раз отправлять авторизованные запросы.
Например, в реализованном веб приложении такой токен сохраняется в локальном хранилище
(localStorage).



\newpage
\section{Технико-экономические показатели}
\subsection{Предполагаемая потребность} ``Клиент-Серверное Android-Приложение
для Управления Скидками в Розничных Сетях'' может быть использована в
потребительской сфере. Программу могут использовать как люди, нуждающиеся в
экономии средств посредством покупки более дешёвых товаров, так и люди,
желающие эффективно и быстро работать со списками покупок.

\subsection{Экономические преимущества разработки} На момент принятия решения о
написании данного продукта во Всемирной сети Интерент был лишь 1 аналог:
приложение ``Едадил''.  Преимущества данной разработки перед конкурентом:
\begin{my_enumerate}
\item Лёгкий, не перегруженный излишним функционалом дизайн
\item Лучший сбор картинок (при помощи кроулинга)
\item Возможность создания неограниченного количества списков покупок с разными названиями
\item Возможность добавления в списки покупок товаров, которых нет в наличии в магазинах
\end{my_enumerate}


\newpage
\section{Источники, используемые при разработке}
\subsection{Список используемой литературы}
\begin{my_enumerate}
    \item
    ГОСТ 19.102-77 Стадии разработки. //Единая система программной документации. -М.: ИПК Издательство стандартов, 2001.
    
    \item
    ГОСТ 19.201-78 Техническое задание. Требования к содержанию и оформлению // Единая система программной документации. -М.:ИПК Издательство стандартов, 2001.
    
    \item  ГОСТ 19.404-79 Пояснительная записка. Требования к содержанию и оформлению. //Единая система программной документации. – М.: ИПК Издательство стандартов, 2001
    
    \item
    ГОСТ 19.101-77 Виды программ и программных документов
    //Единая система программной документации. -М.: ИПК Издательство стандартов, 2.: 001.
    
    \item
    ГОСТ 19.103-77 Обозначения программ и программных документов. //Единая система программной документации. -М.: ИПК Издательство стандартов, 2001.
    
    \item
    ГОСТ 19.104-78 Основные надписи //Единая система программной документации. -М.: ИПК Издательство стандартов, 2001.
    
    \item 
    ГОСТ 19.106-78 Требования к программным документам, выполненным печатным способом. //Единая
    система программной документации. – М.: ИПК Издательство стандартов, 2001
    
    \item 
    ГОСТ 19.603-78 Общие правила внесения изменений. //Единая система программной документации. –
    М.: ИПК Издательство стандартов, 2001
    
    \item
    Oculus Documentation [Электронный ресурс]: Режим доступа: https:// developer3.oculus. com/documentation/
    
    \item
    Oculus Developers Blog [Электронный ресурс]: chrispruett – Squeezing Performance out of your Unity Gear VR Game, 2015 - Режим доступа: https://developer3.oculus.com/blog /squeezing-performance-out-of-your-unity-gear-vr-game/
    
    \item 
    Uninty Scripting Reference [Электронный ресурс]: Режим доступа: https: //docs.unity3d. com/ScriptReference/
    
\end{my_enumerate}



\newpage
\section{Приложение 1. Терминология}
\subsection{Терминология}
\begin{description}

\item[Пользовательская сессия, user session] -- 
    Начинается с момента входа пользователя в систему (log in), и завершается при выходе (log out).
\item[Responsive web design] -- дизайн веб-страниц, обеспечивающий правильное отображение сайта на 
различных устройствах, подключённых к интернету и динамически подстраивающийся под заданные размеры 
окна браузера. \cite{wiki_adaptive};
\end{description}



\newpage
\section{Приложение 2. Описание классов}
Ниже приведены описания классов для Андроид клиента (Java)

Классы и интерфейсы Java:

\subsection{LoginActivity}
Поля:\\
\begin{small}
    \begin{verbatim}
    private EditText etUsername;
    private EditText etPassword;
    private TextView tvRegister, tvHack;
    private Button btnLogin;
    private Button.OnClickListener btnLoginListener;
    private SharedPreferences sharedPrefs;
    private Map<String, String> userData;
    \end{verbatim}
\end{small}
Класс, отвечающий за отображение LoginActivity, проверку введённых данных и отправку запроса авторизации на сервер.

\subsection{MainActivity}
Поля:\\
\begin{small}
    \begin{verbatim}
    private boolean doubleBackToExitPressedOnce = false;
    private SwipeRefreshLayout swipeRefreshLayout;
    private static final String TAG_FRAGMENT_ONE = "fragment_one";
    private static final String TAG_FRAGMENT_TWO = "fragment_two";
    private Shop selectedShop = null;
    private View btnAdd;
    private View btnLoginLogout;
    public boolean homeActive = true;
    public int currentPage = 1;
    public String selectedCategory = "";
    public int totalItemsCount;
    public Parcelable itemsFragmentState;
    public Parcelable shopListsPreviewFragmentState;
    public ItemAdapter adapter;
    public ShopListsPreviewAdapter shopListsPreviewAdapter;
    public FetchData fetchData;
    public List<Shop> shops;
    public RequestQueue queue;
    private FragmentManager fragmentManager;
    public List<String> categories;
    private int currentFragmentId;
    \end{verbatim}
\end{small}
Основной класс программы, хранит состояния фрагментов, чтобы состояния прокрутки оставалась неизменной при переходе по фрагментам; запускает процессы отображения всех основных графических элементов, отвечает за логику перемещения внутри приложения и определяет поведение кнопки ``назад''.
\subsection{RegisterActivity}
Поля:\\
\begin{small}
    \begin{verbatim}
    private EditText etUsername, etPassword;
    private Button btnRegister;
    private Map<String, String> userData;
    \end{verbatim}
\end{small}
Класс, отвечающий за отображение RegisterActivity, проверку введённых данных и отправку запроса авторизации на сервер.

\subsection{ShopListActivity}
Поля:\\
\begin{small}
    \begin{verbatim}
    RecyclerView.LayoutManager layoutManager;
    RecyclerView rvShopList;
    SwipeRefreshLayout swipeRefreshLayout;
    public ItemAdapter adapter;
    public ShopList selectedShopList;
    public FetchData fetchData;
    public TextView tvTotalPrice;
    private View btnAdd;
    private TextView tvShopListName;
    \end{verbatim}
\end{small}
Класс, отвечающий за отображение ShopListActivity, загрузку списков покупок c серверa.


\subsection{ItemAdapter}
Поля:\\
\begin{small}
    \begin{verbatim}
    private List<Item> items;
    private List<Item> tmpItems;
    private ArrayList<Item> itemsCopy;
    private Context context;
    private Dialog itemFullPreview;
    private SparseBooleanArray expandState;
    private boolean type1Downloaded = false;
    private boolean type2Downloaded = false;
        \end{verbatim}
\end{small}
Класс, отвечающий за отображение каждого индивидуального Item'a.

\subsection{ShopListPreviewAdapter}
 Поля:\\
\begin{small}
    \begin{verbatim}
    public List<ShopList> shopLists;
    private Context context;
    \end{verbatim}
\end{small}
Класс, отвечающий за отображение каждого индивидуального ShopList'a.

\subsection{HomeFragment}
Поля:\\
\begin{small}
    \begin{verbatim}
    private RecyclerView rvItemList;
    private RecyclerView.LayoutManager layoutManager;
    \end{verbatim}
\end{small}
Класс, отвечающий за отрисовку главного фрагмента и toolbar'a.


\subsection{ShopListsPreviewFragment}
Поля:\\
\begin{small}
    \begin{verbatim}
    RecyclerView.LayoutManager layoutManager;
    RecyclerView rvShopLists;
    \end{verbatim}
\end{small}
Отображает ``превью'' всех пользовательских предметов и товаров из магазина в
виде карточек


\subsection{InternetUtil}
Класс, содержащий статический метод для проверки работы интернет-сети.

\subsection{JSONUtil}
Класс, формирующий хэдер JSON запроса.

\subsection{APIRequests}
Имеет статическую фабрику для получения Request Handler'a. Далее все действия
проводит RH. он добавляет к запросу URL, ReqponceListener и ErrorListener

\subsection{Config}
Поля:\\
\begin{small}
    \begin{verbatim}
    public static final String URL_CORE = "http://gcsales.ru/";
    public static final String URL_LOGIN = URL_CORE + "auth/login/";
    public static final String URL_REGISTER = URL_CORE + "auth/register/";
    public static final String URL_SALES_SHOP = URL_CORE + "api/shops/";
    public static final String URL_SHOPLIST = URL_CORE + "api/shoplist/";
    public static final String URL_SHOPLISTS_PREVIEW = URL_CORE + "api/shoplist?mode=preview";
    public static final String URL_SHOPLISTS = URL_CORE + "api/shoplist?mode=full";
    public static final String URL_ITEMS_ON_PAGE = "&page=";
    public static final String URL_ITEMS_IN_CATEGRY = "?category=";
    public static final String URL_CATEGORIES = "categories";
    public static final String URL_SL_ADD_ITEM = "additem?id=";
    public static final String URL_SL_ADD_CUSTOM_ITEM = "additem?custom=";
    public static final String URL_SL_DELETE_ITEM = "deleteitem?id=";
    public static final String KEY_USERNAME = "username";
    public static final String KEY_PASSWORD = "password";
    public static final String TAG_VOLLEY_ERROR = "VOLLEY";
    public static final String REQUESTS_CONTENT_TYPE = "application/json; charset=utf-8";
    public static final String BAD_API_AUTH_RESPONSE = "Wrong username/password.";
    public static final String SH_PREFS_NAME = "easy_sales_token";
    public static final String KEY_TOKEN = "user_token";
    public static final String DEF_NO_TOKEN = "NULL";
    public static final String KEY_ITEMS = "all_items";
    public static final String KEY_SHOPLIST_ITEMS = "shoplist_items";
    public static final String KEY_CURRENT_SHOPLIST = "com.hes.easysales.easysales.ShopList.";
    public static final int MAX_LEN_PREVIEW = 20;
    public static final int MAX_ITEMS_PREVIEW = 6;
\end{verbatim}
\end{small}
Класс с общими для всех даными.

\subsection{EndlessRCVScrollListener}
Поля:\\
\begin{small}
    \begin{verbatim}
    private LinearLayoutManager linearLayoutManager;
    private WeakReference<Activity> actvityRef;
\end{verbatim}
\end{small}
Класс, ``слушающий'' когда элемент RecyclerView дойдёт долистается конца. И подгружает ещё больше товаров.

\subsection{FetchData}
Поля:\\
\begin{small}
    \begin{verbatim}
    private ProgressBar pbLoading;

    // To prevent the leak of Context.
    //
    private WeakReference<Activity> activityRef;
    private WeakReference<SwipeRefreshLayout> swipeRefreshLayoutRef;
\end{verbatim}
\end{small}

Класс, который по мере возможности загружает всю необходимтую информацию.o

\subsection{Item}
Поля:\\
\begin{small}
    \begin{verbatim}
    private long id;
    private String name;
    private String category;
    private String imageUrl;
    private double oldPrice;
    private double newPrice;
    private String discount;
    private String dateIn;
    private String dateOut;
    private String condition;
    private Shop shop;
    // These fields are initialized explicitly using setters!
    // Factory method keeps them default.
    // However, Parcel keeps them for future reuse.
    //
    private boolean expandable = false;
    private boolean matched = false;
    private List<Item> matchingItems = new ArrayList<>();
\end{verbatim}
\end{small}

Класс, реализуюций POJO item Item.


\subsection{Shop}
Поля:\\
\begin{small}
    \begin{verbatim}
    private int id;
    private String alias;
    private String name;

    private List<String> categories;
\end{verbatim}
\end{small}

Класс, реализуюций POJO item Shop.


\subsection{ShopList}
Поля:\\
\begin{small}
    \begin{verbatim}
    private long id;
    private String name;
    private List<Item> items;
    private List<Item> customItems;
\end{verbatim}
\end{small}

Класс, реализуюций POJO item ShopList.


\subsection{ItemClickListener}
Метод:\\
\begin{small}
    \begin{verbatim}
    public void onClick(View v, int postition, boolean isLongClick);
\end{verbatim}
\end{small}
Интерфейс, определяющий поведение Item'ов при нажатии на них.



% Index
\newpage

\eskdListOfChanges

% \phantomsection
% \addcontentsline{toc}{section}{Алфавитный указатель}
% \printindex

\end{document}
