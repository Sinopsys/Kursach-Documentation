В данном программном документе приведена пояснительная записка к программе 
``Клиент-Серверное Android-Приложение для Управления Скидками в Розничных Сетях''.
В данном программном документе, в разделе «Введение» указано наименование
программы, и документы, на основании которых ведется разработка.
В разделе «Назначение и область применения» указано функциональное назначение
программы и краткая характеристика области применения программы.
В данном программном документе, в разделе «Технические характеристики»
содержатся следующие подразделы:
\begin{itemize}
    \item Постановка задачи на разработку программы;
    \item Описание алгоритма и функционирования программы с обоснованием выбора 
    схемы алгоритма решения задачи и возможные взаимодействия программы с 
    другими программами;
    \item Описание и обоснование выбора состава технических и программных 
    средств
\end{itemize}
Так же к документу прикреплены приложения. 
Настоящий документ разработан в соответствии с требованиями:\\
1) ГОСТ 19.101-77 Виды программ и программных документов \cite{gost_types_of_software};\\
2) ГОСТ 19.102-77 Стадии разработки \cite{gost_stages_of_devel};\\
3) ГОСТ 19.103-77 Обозначения программ и программных документов \cite{gost_marking_software};\\
4) ГОСТ 19.104-78 Основные надписи \cite{gost_main_signs};\\
5) ГОСТ 19.105-78 Требования к программным документам \cite{gost_demands_for_docs};\\
6) ГОСТ 19.404-79 Пояснительная записка. Требования к содержанию и оформлению  \cite{gost_pz}.\\


Изменения к данной Пояснительной Записке оформляются согласно ГОСТ 19.603-78 \cite{gost_main_rules_change}.
Перед прочтением данного документа рекомендуется ознакомиться с терминологией,
приведенной в Приложении 1 настоящей Пояснительной Записки.
