\subsection{Терминология}
\begin{description}

  \item[REST (от англ. Representational State Transfer)] -- способ сетевого
    взаимодействия.  REST API определяет набор функций, к которым разработчики
    могут совершать запросы и получать ответы. Взаимодействие происходит по
    протоколу HTTP.  Преимуществом такого подхода является широкое
    распространение протокола HTTP, поэтому REST API можно использовать
    практически из любого языка программирования.

  \item[REST endpoint] -- URL, по которому выполняется запрос к REST API.

  \item[SPA (Single Page Application)] -- это веб-приложение или веб-сайт,
    использующий единственный HTML-документ как оболочку для всех веб-страниц и
    организующий взаимодействие с пользователем через динамически подгружаемые
    HTML, CSS, JavaScript, обычно посредством AJAX.

  \item[AJAX] -- подход к построению интерактивных пользовательских интерфейсов
    веб-приложений, заключающийся в <<фоновом>> обмене данными браузера с
    веб-сервером. В результате, при обновлении данных веб-страница не
    перезагружается полностью, и веб-приложения становятся быстрее и удобнее. 

  \item[DOM] -- это независящий от платформы и языка программный интерфейс,
    позволяющий программам и скриптам получить доступ к содержимому HTML-,
    XHTML- и XML-документов, а также изменять содержимое, структуру и
    оформление таких документов.
    
  \item[ORM (Objec-Relational mapping)] --
    технология программирования, которая связывает базы
    данных с концепциями объектно-ориентированных языков программирования,
    создавая «виртуальную объектную базу данных».
\end{description}
