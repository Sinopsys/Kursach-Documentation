\subsection{Терминология}
\begin{description}


\item[VR] - 
Virtual Reality (Виртуальная Реальность). Созданный программными и техническими 
средствами мир, передаваемый человеку при помощи взаимодействия специальных 
внешних устройств с его органами чувств. 

\item[Google Cardboard] - 
Эксперимент компании Google в области виртуальной реальности, в основе которого 
лежит шлем, который, по замыслу разработчиков, можно собрать из подручных 
материалов. Состоит за картона, оптических линз, липучек-застёжек. Так же 
необходимо наличие смартфона с поддержкой технологии VR и установленным VR 
приложением. Он закрепляется непосредственно в шлеме, а шлем крепится к голове 
пользователя, что передает программе движения говолы. 

\item[VR Mode] - 
Режим отображения картинки на экране мобильного устройства, при котором экран 
разделен на 2 части, на которые выводятся изображения для левого и правого 
глаза. Система линз в Google Cardboard (Cardboard) корректирует геометрию 
изображения, доставляя пользователю полное ощущение присутствия в виртуальном 
3D мире.

\item[Normal Mode] - 
Режим отображения картинки на экране мобильного устройства, при котором экран 
не разделяется на 2 части. Не требует наличия Google Cardboard. 

\item[Reticle] - 
Указатель (прицел) в центре экрана, меняющий свою позицию синхронно с главной 
камерой. Позволяет более точно прицеливаться для взаимодействия с объектами.

\item[Draw Calls] - 
Вызовы отрисовки. То, сколькло объектов отрисовываются на экране за один кадр 
(frame). 

\item[Escape the room] - 
(рус. Выйти из комнаты; покинуть комнату) — жанр компьютерных игр, поджанр 
квестов, основная цель которого - найти выход из запертого помещения, используя 
любые подручные средства. 

\item[Rigidbody] - 
Rigidbody дает игровым объектам физические свойства, такие как масса, влияние гравитации и constraints (ограничения, задержки) по всем осям для задержания объекта на месте.

\item[Collider] - 
Компонент коллайдер определяет фигуру объекта с целью установки физических столкновений.
\end{description}

