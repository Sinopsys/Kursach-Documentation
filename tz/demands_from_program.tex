\subsection{Требования к функциональным характеристикам}

\subsubsection{Состав выполняемых функций. Клиентская часть (Android приложение).}
\begin{my_enumerate}
\item Возможность просмотра списка доступных магазинов с акционными товарами
\item Представление текущих акций для конкретного магазина:
    \begin{my_enumerate}
    \item В виде общего списка
    \item По категориям
    \end{my_enumerate}
\item Поиск товаров по названию
\item Составление списка покупок
\item Редактирование списка покупок
\item Возможность отправки списка покупок другим пользователям
\item Отображение всплывающих подсказок при долгом нажатии на элементы управления button
\item Рекомендация акционных товаров в соответствии со списком покупок
\item Отображение ближайших магазинов на карте
\item При отсутствии интернет-соединения отображать сообщение о ошибке
\item Отображение полосы загрузки
\item Реализация обучающего фрагмента при первом запуске, содержащего руководство пользователя по управлению программой
\end{my_enumerate}

\subsubsection{Состав выполняемых функций. Серверная часть.}
\begin{my_enumerate}
\item Crawling веб-страниц для сбора актуальной информации об акционных товарах
\item Запись акционных товаров во всех магахинах в JSON файл (1 файл на 1 магазин)
\item Доступ к файлам на Apache Server для клиентов
\item Построение графиков по составленной базе данных
\item Релизация панели администратора для управления web-crawler'ом
\end{my_enumerate}

\subsubsection{Требования к временным характеристикам}
При скорости интернет соединения 30Мбит/с:
\begin{my_enumerate}
\item Загрузка одной страницы с товарами -- не более 6 секунд
\item Отправка списка покупок на сервер -- не более 3 секунд
\end{my_enumerate}

\subsubsection{Требования к интерфейсу}
\begin{my_enumerate}
\item Совместимость с графической подсистемой ОС Android {\textregistered}
\item Оформление программы в стиле соответствующему guideline от Google: \url{http://material.io/guidelines/style/color.html}
\item Интуитивная ясность конечному пользователю без наличия специального или профессионального образования
\item Полоса загрузки внизу экрана для индикции состояния скачивания данных с сервера
\end{my_enumerate}

\subsection{Требования к надежности}
\subsubsection{Обеспечение устойчивого функционирования программы}

Для надежной работы программы требуется исполнение следующих требований:
\begin{my_enumerate}
\item Обеспечение поддержания заряда аккумуляторной батареи устройства на
уровне не ниже 50\%, иначе обеспечить бесперебойную подзарядку оборудования
\item Обеспечение использования лицензионного программного обеспечения
\item Обеспечение защиты операционной системы и технических средств от
вредоносного воздействия шпионских программ, компьютерных вирусов и сетевых
червей
\item Обеспечение своевременного обновления программных составляющих мобильного устройства
\item При изменении дизайна веб-сайта магазина, администратор приложения оперативно
исправляет соответствующие селекторы для кроулера
\item Раз в сутки производится бэкап всех баз данных
\end{my_enumerate}


\subsubsection{Время восстановления после отказа} 
В случае возникновения сбоя, вызванного внешними факторами (непредвиденное
выключение питания, устранимые неполадки оборудования) время восстановления
программы не должно превышать суммарного затраченного времени на решение
проблем с используемым мобильным устройством и его перезагрузки. Если программа
была аварийно завершена в связи с некорректными действиями оператора, то время
восстановления программы не должно превышать времени ее повторного запуска.

\subsubsection{Отказы из-за некорректных действий оператора}
В случае установки программы на устройство, не имеющего необходимых технических
характеристик, пользователю должно сообщаться об ошибке


\subsection{Условия эксплуатации}
Пользователь данного программного продукта должен разбираться в работе мобильных
устройств, уметь устанавливать и удалять программы, запускать их. Перед использова-
нием программы пользователь должен заранее проинструктирован и уведомлен о соста-
ве выполняемых функций и других характеристиках приложения.

\subsection{Требования к составу и параметрам технических средств}
Для оптимальной работы приложения необходимо учесть следующие системные требо-
вания:
\begin{my_enumerate}
    \item Мобильный телефон со следующими минимальными требованиями:
        \begin{my_enumerate}
            \item Операционная система Android версии 4.4.4 KitKat и выше (API level 19+)
            \item 64-разрядный (x64) процессор
            \item 1ГБ оперативной памяти (ОЗУ)
            \item 100 МБ свободного места на внутреннем накопителе
        \end{my_enumerate}
\end{my_enumerate}

\subsection{Требования к информационной и программной совместимости}
В качестве среды разработки используется среда Android Studio версии
3.0.0.18-1, версия google android SDK 26.1.1-1 и язык Java версии 8.
Для написания web-crawler'a используется язык Python версии 3.6 и 
фреймворк Scrapy.


\subsection{Требования к маркировке и упаковке}
Программа поставляется в виде установочного .apk файла на внешнем носителе
информации – CD/DVD диске. На нем также должны содержаться программная
документация и презентация проекта.

\subsection{Требования к транспортированию и хранению}
Особые требования к транспортировке и хранению не предъявляются.

% EOF

