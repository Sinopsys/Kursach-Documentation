\subsection{Требования к функциональным характеристикам}

\subsubsection{Состав выполняемых функций. Клиентская часть.}
\begin{my_enumerate}
\item Представление текущих акций для конкретного магазина:
    \begin{my_enumerate}
    \item В виде общего списка
    \item По категориям
    \end{my_enumerate}
\item Поиск товаров по названию
\item Составление списка покупок
\item Редактирование списка покупок
\item Синхронизация сессий пользователей на разных устройствах
\item Возможность отправки списка покупок другим пользователям
\item Рекомендация акционных товаров в соответствии со списком покупок
\item Возможность просмотра списка доступных магазинов с акционными товарами
\item Отображение ближайших магазинов на карте
\end{my_enumerate}

\subsubsection{Состав выполняемых функций. Серверная часть.}
\begin{my_enumerate}
  \item 
  Реализация RESTful API:
  \begin{my_enumerate}
    \item Обработка клиентских запросов
    \item Отправка ответов клиентам
  \end{my_enumerate}
  \item Проектирование базы данных
  \begin{my_enumerate}
    \item Хранение информации в базе данных 
    \item Извлечение информации из базы данных
  \end{my_enumerate}
\end{my_enumerate}

\subsubsection{Требования к хранению данных}
Данные об учетных записях пользователей и их списках покупок,
о всех акционных товарах за предшествующий период,
а также о текущих акциях
хранятся в базе данных.

\subsubsection{Требования к протоколу передачи данных}
Коммуникация между клиентом и сервером происходит посредством обмена данными в формате JSON.

\subsubsection{Требования к временным характеристикам}
При скорости интернет соединения 30 Мбит/с
\begin{my_enumerate}
\item Получение любого ответа от сервера -- не более 6 секунд
\item Отправка любого запроса на сервер -- не более 3 секунд
\end{my_enumerate}

\subsubsection{Требования к интерфейсу}
Интерфейс приложения должен подчиняться стилю \textit{``responsive web design''}.\\
Интерфейс содержит следующие компоненты:
\begin{my_enumerate}
  \item Меню навигации (присутствует на всех страницах):
  \begin{my_enumerate}
    \item Кнопка ``Акции''
    \item Кнопка ``Список покупок''
    \item Кнопка ``Корзина''
    \item Кнопка ``Статистика''
    \item Кнопка ``Вход/регистрация''
  \end{my_enumerate}
  \item Страница ``Акции'':
  \begin{my_enumerate}
    \item Меню выбора способа отображения:
    \begin{my_enumerate}
      \item По категориям
      \item Общим списком
    \end{my_enumerate}
    \item Карточки с продуктами, каждая из которых содержит поле:
    \begin{my_enumerate}
      \item Название товара
      \item Категория товара
      \item Старая цена на товар
      \item Новая цена на товар
      \item Скидка в процентах
      \item Дата проведения акции
      \item Специальные условия
    \end{my_enumerate}
    \item Пагинация
  \end{my_enumerate}
  \item Страница ``Список покупок''
  \begin{my_enumerate}
    \item Поле ``Общая сумма''
    \item Поле ``Экономия''
    \item Список с добавленными товарами
    \item Кнопка ``Добавить новую позицию''
  \end{my_enumerate}
  \item Страница ``Статистика''
  \item Страница ``Карта''
\end{my_enumerate}


\subsection{Требования к надежности}

\subsubsection{Обеспечение устойчивого функционирования программы}
Для надежной работы программы требуется исполнение следующих требований:
\begin{my_enumerate}
\item При изменении дизайна веб-сайта магазина, администратор приложения оперативно
исправляет соответствующие селекторы для кроулера
\item Персональные данные пользователей должны храниться в зашифрованном виде
\item Раз в сутки производится бэкап всех баз данных
\item Ширина канала на сервере - 100Мбит
\end{my_enumerate}

\subsubsection{Время восстановления после отказа}
В случае непридвиденного завершения работы приложения или сбоя, вызванного
внешними факторами, время восстановления не должно превышать суммарного времени,
затраченного на восстановление работоспособности и программы и повторной загрузки веб-страницы.
Если программа была аварийно завершена в связи с некорректными действиями оператора, 
то время восстанов ления программы не должно превышать времени повторной загрузки веб-страницы.

\subsubsection{Отказы из-за некорректных действий оператора}
В случае установки программы на устройство, не имеющего необходимых технических
характеристик, пользователю должно сообщаться об ошибке


\subsection{Условия эксплуатации}
Пользователь должен обладать базовыми навыками управления веб-браузером
на ПК или мобльном устройстве.


\subsection{Требования к составу и параметрам технических средств}
Необходимо наличие ПК или мобильного устройства со стабильным интернет соединением.\\
Для корректной работы приложения необходим один из следующих веб-браузеров \cite{css_grid}:
\begin{my_enumerate}
  \item Firefox версии 52 и выше
  \item Google Chrome версии 50 и выше
  \item Safari версии 10.1 и выше
  \item Opera версии 48 и выше
\end{my_enumerate}


\subsection{Требования к информационной и программной совместимости}
Разработка ведется с использованием следующих технологий:
\begin{my_enumerate}
  \item Язык программирования JavaScript стандарта ECMAScript 5
  \item Язык разметки HTML версии 5
  \item Таблицы стилей CSS
  \item Фреймворк Vue.js
  \item СУБД MySql
  \item Язык программирования Java версии 8
\end{my_enumerate}


\subsection{Требования к маркировке и упаковке}
Приложение распространяется по веб-адресу, который будет уточнен на момент релиза.
В комплект поставки программы также входит USB флеш-накопитель или CD/DVD - диск,
который содержит техническую документацию и презентацию проекта. 


\subsection{Требования к транспортированию и хранению}
Требования к транспортированию и хранению не предъявляются.


\subsection{Специальные требования}
Специальные требования не предъявляются.
