\subsection{Требования к функциональным характеристикам}

\subsubsection{Состав выполняемых функций. Серверная часть.}
\begin{my_enumerate}
  \item REST API:
    \begin{my_enumerate}
      \item Обработка запросов для работы со списками покупок пользователей
      \item Обработка запросов для кроулера
      \item Обработка запросов для получения списка актуальных акционных
        товаров
      \item Обработка запросов для авторизации пользователей
    \end{my_enumerate}
  \item База данных для хранения:
    \begin{my_enumerate}
      \item Акционных товаров
      \item Аккаунтов пользователей
      \item Списков покупок пользователей
    \end{my_enumerate}
\end{my_enumerate}

\subsubsection{Состав выполняемых функций. Клиентская часть.}
\begin{my_enumerate}
  \item Возможность просмотра списка доступных магазинов с акционными товарами
  \item Представление текущих акций для конкретного магазина:
    \begin{my_enumerate}
      \item В виде общего списка
      \item По категориям
    \end{my_enumerate}
  \item Возможность регистрации и авторизации пользователей в системе для
    работы со списком покупок
  \item Создание и удаление списка покупок
  \item Работа со списком покупок:
    \begin{my_enumerate}
      \item Добавление и удаление товаров магазинов
      \item Добавление и удаление пользовательских позиций
      \item Рекомендация товаров магазинов на основе пользовательских позиций 
    \end{my_enumerate}
\end{my_enumerate}

\subsubsection{Требования к хранению данных}
Данные об учетных записях пользователей и их списках покупок,
о всех акционных товарах за предшествующий период,
а также о текущих акциях
хранятся в базе данных.

\subsubsection{Требования к протоколу передачи данных}
Коммуникация между клиентом и сервером происходит посредством обмена данными в формате JSON.

\subsubsection{Требования к временным характеристикам}
При скорости интернет соединения 30 Мбит/с
\begin{my_enumerate}
\item Получение любого ответа от сервера -- не более 6 секунд
\item Отправка любого запроса на сервер -- не более 3 секунд
\end{my_enumerate}

\subsubsection{Требования к интерфейсу}
\begin{my_enumerate}
  \item Интерфейс приложения должен подчиняться стилю \textit{``responsive web design''}.\cite{wiki_adaptive}
  \item Интерфейс должен быть интуитивно понятен рядовому пользователю без
    специального или профессионального бразования.
\end{my_enumerate}

\subsection{Требования к надежности}

\subsubsection{Обеспечение устойчивого функционирования программы}
Для надежной работы программы требуется исполнение следующих требований:
\begin{my_enumerate}
\item При изменении дизайна веб-сайта магазина, администратор приложения оперативно
исправляет соответствующие селекторы для кроулера
\item Персональные данные пользователей должны храниться в зашифрованном виде
\item Раз в сутки производится бэкап всех баз данных
\item Ширина канала на сервере - 100Мбит
\end{my_enumerate}

\subsubsection{Время восстановления после отказа}
В случае непридвиденного завершения работы приложения или сбоя, вызванного
внешними факторами, время восстановления не должно превышать суммарного времени,
затраченного на восстановление работоспособности и программы и повторной загрузки веб-страницы.
Если программа была аварийно завершена в связи с некорректными действиями оператора, 
то время восстанов ления программы не должно превышать времени повторной загрузки веб-страницы.

\subsubsection{Отказы из-за некорректных действий оператора}
При возникновении исключительных ситуаций из-за некоррентных действий
пользователя должно выводиться сообщение об ошибке.

\subsection{Условия эксплуатации}
Пользователь должен обладать базовыми навыками управления веб-браузером
на ПК или мобльном устройстве.

\subsection{Требования к составу и параметрам технических средств}
Необходимо наличие ПК или мобильного устройства со стабильным интернет соединением.\\
Для корректной работы приложения необходим один из следующих веб-браузеров \cite{css_grid}:
\begin{my_enumerate}
  \item Интернет-браузер Google Chrome версии 49 и выше или
  \item Интернет-бразуер Firefox версии 44 и выше
\end{my_enumerate}

\subsection{Требования к информационной и программной совместимости}
Разработка ведется с использованием следующих технологий:
\begin{my_enumerate}
  \item Язык программирования JavaScript стандарта ECMAScript 5
  \item Язык разметки HTML версии 5
  \item Таблицы стилей CSS
  \item Фреймворк VueJS
  \item СУБД Postgresql
  \item Boostrap 4
  \item NodeJS 8
\end{my_enumerate}

\subsection{Требования к маркировке и упаковке}
Приложение распространяется по веб-адресу, который будет уточнен на момент релиза.
В комплект поставки программы также входит USB флеш-накопитель или CD/DVD - диск,
который содержит техническую документацию и презентацию проекта. 

\subsection{Требования к транспортированию и хранению}
Требования к транспортированию и хранению не предъявляются.

\subsection{Специальные требования}
Специальные требования не предъявляются.

