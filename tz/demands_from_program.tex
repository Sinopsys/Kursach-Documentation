\subsection{Требования к функциональным характеристикам}

\subsubsection{Состав выполняемых функций. Клиентская часть.}
\begin{my_enumerate}
\item Представление текущих акций для конкретного магазина:
    \begin{my_enumerate}
    \item В виде общего списка
    \item По категориям
    \end{my_enumerate}
\item Поиск товаров по названию
\item Составление списка покупок
\item Редактирование списка покупок
\item Синхронизация сессий пользователей на разных устройствах
\item Возможность отправки списка покупок другим пользователям
\item Рекомендация акционных товаров в соответствии со списком покупок
\item Возможность просмотра списка доступных магазинов с акционными товарами
\item Отображение ближайших магазинов на карте
\end{my_enumerate}

\subsubsection{Состав выполняемых функций. Серверная часть.}
\begin{my_enumerate}
\item Запись информации об акционных товарах в темпоральную базу данных
\item Построение графиков по составленной базе данных
\item Хранение пользовательских данных:
    \begin{my_enumerate}
        \item Логинов, паролей в зашифрованном виде
        \item Списка покупок 
        \item Истории покупок
    \end{my_enumerate}
\end{my_enumerate}

\subsection{Требования к интерфейсу}
Интерфейс приложения должен подчиняться стилю \textit{``responsive web design''}.\\
Интерфейс содержит следующие компоненты:
\begin{my_enumerate}
  \item Меню навигации (присутствует на всех страницах):
  \begin{my_enumerate}
    \item Кнопка ``Акции''
    \item Кнопка ``Список покупок''
    \item Кнопка ``Корзина''
    \item Кнопка ``Статистика''
    \item Кнопка ``Вход/регистрация''
  \end{my_enumerate}
  \item Страница ``Акции'':
  \begin{my_enumerate}
    \item Меню выбора способа отображения:
    \begin{my_enumerate}
      \item По категориям
      \item Общим списком
    \end{my_enumerate}
    \item Карточки с продуктами, каждая из которых содержит поле:
    \begin{my_enumerate}
      \item Название товара
      \item Категория товара
      \item Старая цена на товар
      \item Новая цена на товар
      \item Скидка в процентах
      \item Дата проведения акции
      \item Специальные условия
    \end{my_enumerate}
    \item Пагинация
  \end{my_enumerate}
  \item Страница ``Список покупок''
  \begin{my_enumerate}
    \item Поле ``Общая сумма''
    \item Поле ``Экономия''
    \item Список с добавленными товарами
    \item Кнопка ``Добавить новую позицию''
  \end{my_enumerate}
  \item Страница ``Статистика''
  \item Страница ``Карта''
\end{my_enumerate}


\subsection{Требования к временным характеристикам}

\subsubsection{При скорости интернет соединения 30 Мбит/с}
\begin{my_enumerate}
\item Загрузка одной страницы с товарами -- не более 6 секунд
\item Отправка списка покупок на сервер -- не более 5 секунд
\item \textbf{TODO}
\end{my_enumerate}


\subsection{Требования к протоколу передачи данных}

Коммуникация между клиентом и сервером происходит посредством обмена данными в формате JSON.

\subsection{Требования к надежности}
\subsubsection{Обеспечение устойчивого функционирования программы}

Для надежной работы программы требуется исполнение следующих требований:
\begin{my_enumerate}
\item При изменении дизайна веб-сайта магазина, администратор приложения оперативно
исправляет соответствующие селекторы для кроулера
\item Персональные данные пользователей должны храниться в зашифрованном виде
\item Раз в сутки должен производиться бэкап всех баз данных
\end{my_enumerate}
\textbf{TODO} Написать про ширину канала \\
