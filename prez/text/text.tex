%!TEX TS-program=xelatex
%!USE flag=shell-escape
\documentclass[12pt]{article}
%%% Работа с русским языком и шрифтами
\usepackage[english,russian]{babel}   % загружает пакет многоязыковой вёрстки
\usepackage{fontspec}      % подготавливает загрузку шрифтов Open Type, True Type и др.
\defaultfontfeatures{Ligatures={TeX},Renderer=Basic}  % свойства шрифтов по умолчанию
\setmainfont[Ligatures={TeX,Historic}]{Myriad Pro} %  установите шрифты Myriad Pro или (при невозможности) замените здесь на другой шрифт, который есть в системе — например, Arial
\setsansfont{Myriad Pro}  %  установите шрифты Myriad Pro или (при невозможности) замените здесь на другой шрифт, который есть в системе — например, Arial
\setmonofont{Courier New}
\usepackage[top=2cm, bottom=2cm, left=2cm, right=2cm]{geometry}

%%%%%% My commands %%%%%%%
%
\newcommand{\kir}{\textbf{Кирилл: }}
\renewcommand{\max}{\textbf{Максим: }}
\renewcommand{\line}{\noindent\rule{\textwidth}{1pt}}
%

\begin{document}

\section*{Текст презентации}

\kir\ Добрый день, уважаемая комиссия, студенты. Сейчас я, Куприянов Кирилл и мой
коллега Суровцев Максим расскажем вам про курсовой проект, который мы
разработали в этом году.\\
\max\ Тема: Клиент-Серверное Приложение для Управления Скидками в Розничных
Сетях, а научный руководитель --- Александров Дмитрий Владимирович.\\
\line\
\kir\ В последнее время люди чаще стали покупать товары по скидкам и акциям. Это
привело к появлению агрегаторов скидок и купонов, и к созданию новой
соответствующей ниши для приложений и программ. Её мы и решили занять, найдя
проблему и обосновав актуальность.\\
\line\
\kir\ В процессе своего выступления мы будем использовать технические термины,
которые стоит сразу обозначить.\\
\max\ REST API, ORM\\
\kir\ \textit{Cralwer} --- это программный модуль,
который, как будет сказано позже, состоит из многих подмодулей. Он работает в
фоне, собирая данные с сайтов и взаимодействует с сервером путем отправки
файлов JSON.\ \textit{Activity, Активность} ---
определение из Андроида. Это графический компонент, содержащий другие
компоненты, и наилучшим сравнением будет --- окно программы.\\
\line\
\max\ Проводя исследование актуальности работы, мы столкнулись с интересной вещью.
Статистика популярности запроса ``скидки'' в интернете имеет сезонность.
Оказывается, зимой люди склонны больше покупать товаров по скидке, чем в другие
времена года. В целом график показывает рост интереса людей к скидкам.\\
\kir\ Второй график иллюстрирует рост популярности крупного аналога ---
приложения ``Едадил'', что говорит о хорошей конкуренции в этой сфере.


% добавить в библиографию гугл тренды. (преза)


\end{document}


% EOF

