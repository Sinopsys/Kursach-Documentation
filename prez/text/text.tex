%!TEX TS-program=xelatex
%!USE flag=shell-escape
\documentclass[12pt]{article}
%%% Работа с русским языком и шрифтами
\usepackage[english,russian]{babel}   % загружает пакет многоязыковой вёрстки
\usepackage{fontspec}      % подготавливает загрузку шрифтов Open Type, True Type и др.
\defaultfontfeatures{Ligatures={TeX},Renderer=Basic}  % свойства шрифтов по умолчанию
\setmainfont[Ligatures={TeX,Historic}]{Myriad Pro} %  установите шрифты Myriad Pro или (при невозможности) замените здесь на другой шрифт, который есть в системе — например, Arial
\setsansfont{Myriad Pro}  %  установите шрифты Myriad Pro или (при невозможности) замените здесь на другой шрифт, который есть в системе — например, Arial
\setmonofont{Courier New}
\usepackage[top=2cm, bottom=2cm, left=2cm, right=2cm]{geometry}

%%%%%% My commands %%%%%%%
%
\newcommand{\kir}{\textbf{Кирилл: }}
\renewcommand{\max}{\textbf{Максим: }}
\renewcommand{\line}{\noindent\rule{\textwidth}{1pt}}
%

\begin{document}

\section*{Текст презентации}

\kir\ Добрый день, уважаемая комиссия, студенты. Сейчас я, Куприянов Кирилл и мой
коллега Суровцев Максим расскажем вам про курсовой проект, который мы
разработали в этом году.\\
\max\ Тема: Клиент-Серверное Приложение для Управления Скидками в Розничных
Сетях, а научный руководитель --- Александров Дмитрий Владимирович.\\
\line\
\kir\ В последнее время люди чаще стали покупать товары по скидкам и акциям. Это
привело к появлению агрегаторов скидок и купонов, и к созданию новой
соответствующей ниши для приложений и программ. Её мы и решили занять, найдя
проблему и обосновав актуальность.\\
\line\
\kir\ В процессе своего выступления мы будем использовать технические термины,
которые стоит сразу обозначить.\\
\max\ REST API, ORM\\
\kir\ \textit{Cralwer} --- это программный модуль,
который, как будет сказано позже, состоит из многих подмодулей. Он работает в
фоне, собирая данные с сайтов и взаимодействует с сервером путем отправки
файлов JSON.\ \textit{Activity, Активность} ---
определение из Андроида. Это графический компонент, содержащий другие
компоненты, и наилучшим сравнением будет --- окно программы.\\
\line\
\max\ Проводя исследование актуальности работы, мы столкнулись с интересной вещью.
Статистика популярности запроса ``скидки'' в интернете имеет сезонность.
Оказывается, зимой люди склонны больше покупать товаров по скидке, чем в другие
времена года. В целом график показывает рост интереса людей к скидкам.\\
\kir\ Второй график иллюстрирует рост популярности крупного аналога ---
приложения ``Едадил'', что говорит о хорошей конкуренции в этой сфере.\\
\line\
\kir\ Проблем мы нашли много. Бумажные каталоги использовать неудобно --- что,
если человек хочек посмотреть скидки, сидя дома на диване? Смотреть сайты
каждого магазина по отдельности времязатратно, а существующие приложения не
обладают достаточным функционалом.\\
\line\
\kir\ Соответстно, мы задались целью создать модульные и масштабируемые web и
Андроид приложения для работы со множеством списков покупок и акциями в
розничных сетях.
\line\
\max\ На момент взятия темы курсовой работы существовал один аналог нашего
приложения --- ``Едадил --- акции в магазинах (бета)''.\\
\line\
\kir\ Перед Едадилом наше приложение выделяется удобством использования --- у
нас предусмотрено неограниченное количество списков покупок, добавление в
списки пользовательских позиций, то есть товаров, которых нет на данный момент
по скидке, и возможность использования одного аккаунта несколькими людьми.  А
такие архитектурные решения как модульность, масштабируемость и
поддерживаемость выделяют нас как программный продукт. Перейдём к обзору
алгоритма работы программы.\\
\line\
\kir\ Работа нашей программы начинается с кроулера. Он собирает данные о товарах с
сайтов магазинов и передаёт серверу. Поскольку кроулер --- модульный, и
рассматривать его лучше по модулям.\\
\line\
\kir\ В модуле selectors были использованы xpath селекторы в качестве
универсального средства поиска элементов в HTML дереве. Например, данный
селектор найдёт все аттрибуты href ссылок с классом {\scriptsize
catalog-categories\_\_link}.\\
\line\
\kir\ Модуль spiders, или пауки, работает следующим образом. Пуак заходит на
сайт магазина, то есть загружает его исходный код. Ищет корневой элемент для
всех товаров, и затем, для каждого товара, создаёт объект класса Item,
заполняет его поля, и передаёт модулю pipelines, про который я расскажу далее.
Спайдер затем переходит на следующую страницу, запускает снова свою процедуру,
и так до конца, пока не кончатся страницы.\\
\line\
\kir\ Каждый объект класса Item поступает на вход модулю pipelines, где одним из
важнейших является класс, записывающий его в Базу Данных путём отправки
POST запроса на сервер.\\
\line\
\max\ ПРО СПИСОК ПОКУПОК\\
\line\
\max\ Пожалуй, главной сущностью в базе данных является item. Именно в таблицу item
записываются данные о товарах, собранных кроулером. Каждый товар содержит
ссылку на магазин, которому он принадлежит.\\
Пользователь может иметь много списков покупок, а список покупок в свою очередь
может содержать много товаров магазинов и много пользовательских позиций. 
Пользовательские позиции из списка покупок отображает таблица custom\_item.\\
\line\
\max\ Для взаимодействия сервера и клиентов был разработан REST API.\\ Список
магазинов, список категорий для магазина у нас не захардкожены на клиентах —
они динамически подгружаются при запуске приложения. Это значит, что если мы
захотим добавить новый магазин, нам достаточно будет сделать insert в базе
данных, написать спайдер с селекторами для него и запустить (без перезагрузки
сервера).\\ На слайде приведен пример endpoint'а. Голубым цветом выделен id
магазина, розовым — параметры запроса. Сервер вернет ответ, содержащий первую
страницу товаров категории ``Напитки'' для магазина с id равным 1.\\ Товары
загружаются постранично для экономии трафика.\\
\line\
\max\ Endpoint'ы для работы со списком покупок должны быть авторизованы. Для
авторизации используются JSON Web Token.\\ Токен состоит из трех частей:
header (красный), payload (розовый) и signature (голубой). Header содержит тип
токена (это JWT) и способ хэширования, используемый для создания сигнатуры.  В
payload'е содержится основная информация — это имя пользоватея, его id и
timestap создания токена. Именно отсюда сервер берет id и имя пользователя.\\
Сигнатура токена получается следующим образом: header и payload кодируются в
формат base64 и конкатенируются через точку. После этого полученная строка
передается в хэш-функцию вместе с секретным ключом.\\
\line\
\max\ Алгоритм авторизации пользователя следующий:\\
1. Клиентское приложение отправляет POST запрос, в теле которого указывает имя
пользователь и пароль, введенные пользователем.\\
2. Если имя пользователя и пароль верны, сервер возвращает токен.\\
3. Клиентское прилоежение добавляет токен в заголовок (header) каждого авторизованного запроса.\\
4. Сервер проверяет сигнатуру токена с помощью секретного ключа, и если она
оказалось валидной возвращает ответ, а иначе возвращает ошибку.\\
\line\
\kir\ В разработке были использованы: Фреймворк Scrapy и язык Python для
создания кроулера и его модулей, PostgreSQL и Sequelize для базы данных и ORM,
и фреймворк express для создания REST API.\ Фреймворки VUE js и bootstrap для
веб-клиента, а под Андроид разработка велась с использованием языка Java.  Для
задания расписания запуска кроулера использовалась стандартная Unix утилита
crontab. Для создания программной документации и презентации, была использована
система вёрстки текста {\LaTeX}.\\
\line\
\kir\ И конечно же, помимо серверной части, были реализованы 2 клиента. Их работу мы продемонстрируем на видео.\\
\line\
\kir\ Поскольку направление разработки достаточно новое и развивающееся, путей развитя может быть множество: от увеличения числа магазинов, до создания клиентах на других популярных платформах\\
\line\
\line\
\kir\ Спасибо за внимание! С удовольствием ответим на ваши вопросы.
\end{document}


% EOF

