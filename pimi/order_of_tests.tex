%=========================================
\subsection{Параметры технических средств, используемых во время испытаний}
Для испытания программы необходимо учесть следующие системные требования:
\begin{my_enumerate}
    \item Мобильный телефон со следующими минимальными требованиями:
    \begin{my_enumerate}
        \item Операционная система Android версии 4.4.4 KitKat и выше (API level 19+)
        \item 64-разрядный (x64) процессор
        \item 1ГБ оперативной памяти (ОЗУ)
        \item 100 МБ свободного места на внутреннем накопителе
    \end{my_enumerate}
    \item Наличие интернет-соединения со скоростью не меньше 30Мбит/c
\end{my_enumerate}


%=========================================
\subsection{Порядок проведения испытаний}
Испытания проводятся поэтапно, друг за другом, в следующем порядке:
\begin{my_enumerate}
    \item Испытание выполнения требований к программной документации
    \item Испытание выполнения требований к графическому интерфейсу и оформлению программы
    \item Испытание выполнения требований к функциональным характеристикам программы, надежности и корректности ее работы
    \item Испытание выполнения требований к временным характеристикам
\end{my_enumerate}


%=========================================
\subsection{Условия проведения испытаний}

\subsubsection{Требования к численности и квалификации персонала}
Минимальное количество персонала, требуемого для работы программы: 1 оператор.
Пользователь данного программного продукта должен разбираться в работе
мобильных устройств, уметь устанавливать и удалять программы, запускать их.
Перед использованием программы пользователь должен заранее проинструктирован и
уведомлен о составе выполняемых функций и других характеристиках приложения.

