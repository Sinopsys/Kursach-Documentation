\subsection{Список используемой литературы}
\begin{my_enumerate}
    \item
    ГОСТ 19.102-77 Стадии разработки. //Единая система программной документации. -М.: ИПК Издательство стандартов, 2001.
    
    \item
    ГОСТ 19.201-78 Техническое задание. Требования к содержанию и оформлению // Единая система программной документации. -М.:ИПК Издательство стандартов, 2001.
    
    \item  ГОСТ 19.404-79 Пояснительная записка. Требования к содержанию и оформлению. //Единая система программной документации. – М.: ИПК Издательство стандартов, 2001
    
    \item
    ГОСТ 19.101-77 Виды программ и программных документов
    //Единая система программной документации. -М.: ИПК Издательство стандартов, 2.: 001.
    
    \item
    ГОСТ 19.103-77 Обозначения программ и программных документов. //Единая система программной документации. -М.: ИПК Издательство стандартов, 2001.
    
    \item
    ГОСТ 19.104-78 Основные надписи //Единая система программной документации. -М.: ИПК Издательство стандартов, 2001.
    
    \item 
    ГОСТ 19.106-78 Требования к программным документам, выполненным печатным способом. //Единая
    система программной документации. – М.: ИПК Издательство стандартов, 2001
    
    \item 
    ГОСТ 19.603-78 Общие правила внесения изменений. //Единая система программной документации. –
    М.: ИПК Издательство стандартов, 2001
    
    \item
    Oculus Documentation [Электронный ресурс]: Режим доступа: https:// developer3.oculus. com/documentation/
    
    \item
    Oculus Developers Blog [Электронный ресурс]: chrispruett – Squeezing Performance out of your Unity Gear VR Game, 2015 - Режим доступа: https://developer3.oculus.com/blog /squeezing-performance-out-of-your-unity-gear-vr-game/
    
    \item 
    Uninty Scripting Reference [Электронный ресурс]: Режим доступа: https: //docs.unity3d. com/ScriptReference/
    
\end{my_enumerate}

