\subsection{Требования к функциональным характеристикам}

\subsubsection{Состав выполняемых функций}
\begin{my_enumerate}
\item Запуск игры в режиме Virtual Reality
\item Запуск игры в режиме Normal Mode
\item Reticle (прицел/указатель) в центре экрана для более точного наведения на цель
\item Возможность передвижения персонажа без нажатия кнопоки
\item Звук ходьбы персонажа, звук взятия предметов, открытия/закрытия двери, дождя
\item Возможность взятия предметов без нажатия кнопки
\item Возможность бросить предмет без нажатия кноки
\item Показ текста пользователю для отображения необходимой на конкретный момент времени информации
\item Логика головоломок, цепочка решений которых приведет к ключу
\item Реализация обучающего фрагмента в начале игры, содержащего руководство пользователя по управлению в игре
\end{my_enumerate}


\subsection{Требования к интерфейсу}

\begin{my_enumerate}
\item Совместимость с графической подсистемой ОС Android®
\item Оформление программы в стиле соответствующему guideline от Google:\\ material.io/guidelines/style/color.html
\item Интуитивная ясность конечному пользователю без наличия специального или профессионального образования
\end{my_enumerate}


\subsection{Требования к временным характеристикам}
\subsubsection{Требования от разработчиков Oculus для VR приложений}
\begin{my_enumerate}
\item 50 - 100 draw calls per frame
\item 50k – 100k polygons per frame
\item As few textures as possible (but they can be large)
\item 1 - 3 ms spent in script execution (Unity Update())
\end{my_enumerate}


\subsection{Требования к надежности}
\subsubsection{Обеспечение устойчивого функционирования программы}
Для надежной работы программы требуется исполнение следующих требований:
\begin{my_enumerate}
\item Обеспечение поддержания заряда аккумуляторной батареи устройства на уровне не ниже 50\%, иначе обеспечить бесперебойную подзарядку оборудования
\item Обеспечение использования лицензионного программного обеспечения
\item Обеспечение защиты операционной системы и технических средств от вредоносного воздействия шпионских программ, компьютерных вирусов и сетевых червей
\item Обеспечение своевременного обновления программных составляющих мобильного устройства
\end{my_enumerate}


\subsection{Время восстановления после отказа}
В случае возникновения сбоя, вызванного внешними факторами (непредвиденное выключение питания, устранимые неполадки оборудования) время восстановления программы не должно превышать суммарного затраченного времени на решение проблем с используемым мобильным устройством и его перезагрузки. Если программа была аварийно завершена в связи с некорректными действиями оператора, то время восстановления программы не должно превышать времени ее повторного запуска.

\subsubsection{Отказы из-за некорректных действий оператора}
В случае установки программы на устройство, не поддерживающее VR, пользователю должно сообщаться об ошибке.


\subsection{Условия эксплуатации}
Пользователь данного программного продукта должен разбираться в работе мобильных устройств, уметь устанавливать и удалять программы, запускать их. Перед использованием программы пользователь должен заранее проинструктирован и уведомлен о составе выполняемых функций и других характеристиках приложения.


\subsection{Требования к составу и параметрам технических средств}
Для оптимальной работы приложения необходимо учесть следующие системные требования:
\begin{my_enumerate}
\item Мобильный телефон со следующими минимальными требованиями:
    \begin{my_enumerate}
   	\item Операционная система Android версии 4.4.4 KitKat и выше (API level 19+)
    \item 64-разрядный (x64) процессор
    \item 1ГБ оперативной памяти (ОЗУ)
    \item 100 МБ свободного места на внутреннем накопителе
    \item Наличие гироскопа
    \item Наличие акселерометра
    \item Размеры устройства не более 165мм х 90 мм
    \item Длина устройства не менее 135мм
    \end{my_enumerate}
\item Любая модель Google Cardboard
\end{my_enumerate}


\subsection{Требования к информационной и программной совместимости}
В качестве среды разработки используется среда Unity5 версии 5.5.0f3.
Разработка графической части производится с использованием набора бесплатных средств Asset Store.
Скрипты для установки поведения объектов и связи между ними, для написания логики всех головоломок и сюжетной линии, пишутся на языке $C\#$ в среде Microsoft Visual Studio 2017. 


\subsection{Требования к упаковке}
Программа поставляется в виде установочного .apk файла на внешнем носителе информации – DVD диске. На нем также должны содержаться программная документация и презентация проекта.
