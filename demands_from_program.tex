\subsection{Требования к функциональным характеристикам}

\subsubsection{Состав выполняемых функций. Клиентская часть.}
\begin{my_enumerate}
\item Представление текущих акций для конкретного магазина:
    \begin{my_enumerate}
    \item В виде общего списка
    \item По категориям
    \end{my_enumerate}
\item Составление списка покупок
\item Редактирование списка покупок
\item Синхронизация сессий пользователей на разных устройствах
\item Возможность отправки списка покупок другим пользователям
\item Рекомендация акционных товаров в соответствии со списком покупок
\item Возможность просмотра списка доступных магазинов с акционными товарами
\item Отображение ближайших магазинов на карте
\end{my_enumerate}

\subsubsection{Состав выполняемых функций. Серверная часть.}
\begin{my_enumerate}
\item Запись информации об акционных товаров в темпоральную базу данных
\item Построение графиков по составленной базе данных
\item Защита данных пользователей
\end{my_enumerate}

\subsection{Требования к временным характеристикам}

\subsubsection{При скорости интернет соединения 30 Мбит/с}
\begin{my_enumerate}
\item Загрузка одной страницы с товарами -- не более 5 секунд
\item Отправка списка покупок на сервер -- не более 5 секунд
\item \textbf{TODO}
\end{my_enumerate}


\subsection{Требования к протоколу передачи данных}

Коммуникация между клиентом и сервером происходит посредством обмена данными в формате JSON.

\subsection{Требования к надежности}
\subsubsection{Обеспечение устойчивого функционирования программы}
\begin{my_enumerate}
\item При изменении дизайна веб-сайта магазина, администратор приложения оперативно
исправляет соответствующие селекторы для кроулера
\item Персональные данные пользователей должны храниться в зашифрованном виде
\item Раз в сутки должен производиться бэкап всех баз данных
\end{my_enumerate}
\textbf{TODO} Написать про ширину канала \\
