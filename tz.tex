\documentclass[encoding=utf8]{twoeskd}


\usepackage[export]{adjustbox}
\usepackage{graphicx}
\usepackage[utf8]{inputenc}
\usepackage{multicol}
\usepackage{multirow}
\usepackage{makeidx}


\usepackage{hyperref}
\hypersetup{
	colorlinks,
	citecolor=black,
	filecolor=black,
	linkcolor=black,
	urlcolor=black
}

\usepackage[backend=biber
           % ,style=authoryear-icomp
            ]{biblatex}
\addbibresource{used_sources.bib}

% NLS support packages
\usepackage[T2A]{fontenc}
\usepackage[russian]{babel}
\usepackage{pscyr}

% Font selection
\usepackage{courier}
\usepackage{amssymb}

\setlength{\parindent}{0cm}
\setlength{\parskip}{0.2cm}

\newcommand\tab[1][1cm]{\hspace*{#1}}

% debug to see the frame borders
% from https://en.wikibooks.org/wiki/LaTeX/Page_Layout
% \usepackage{showframe}

% Indices & bibliography
%\usepackage{natbib}
\usepackage[titles]{tocloft}
\setcounter{tocdepth}{3}
\setcounter{secnumdepth}{5}
\makeindex

\renewcommand{\cftsecleader}{\cftdotfill{\cftdotsep}}

% change style of titles in \section{}
\usepackage{titlesec}
\titleformat{\section}[hang]{\Large\bfseries\center}{\thetitle.}{1em}{}
\titleformat{\subsection}[hang]{\large\normalfont\raggedright}{\thetitle.}{1em}{\underline}
\titleformat{\subsubsection}[hang]{\normalsize\normalfont\raggedright}{\thetitle.}{1pt}{}

% Packages for text layout in normal mode
% \usepackage[parfill]{parskip} % автоматом делает пустые линии между параграфами, там где они есть в тексте
% \usepackage{indentfirst} % indent even in first paragraph
\usepackage{setspace}	 % controls space between lines
\setstretch{1} % space between lines
\setlength\parindent{0.9cm} % size of indent for every paragraph
\usepackage{csquotes}% превратить " " в красивые двойные кавычки
\MakeOuterQuote{"}


% this makes items spacing single-spaced in enumerations.
\newenvironment{my_enumerate}{
	\begin{enumerate}
		\setlength{\itemsep}{1pt}
		\setlength{\parskip}{0pt}
		\setlength{\parsep}{0pt}}{\end{enumerate}
}
\usepackage{pbox}

% configure eskd
\titleTop{
	{\Large ПРАВИТЕЛЬСТВО РОССИЙСКОЙ ФЕДЕРАЦИИ \\
		НАЦИОНАЛЬНЫЙ ИССЛЕДОВАТЕЛЬСКИЙ УНИВЕРСИТЕТ \\
		«ВЫСШАЯ ШКОЛА ЭКОНОМИКИ» } \\
	\vspace*{0.2cm}
	{Факультет компьютерных наук \\
		Департамент программнoй инженерии \\
	}
}
\titleDesignedByNoYear{Студенты группы БПИ 151 НИУ ВШЭ}{Куприянов К. И.}
\titleDesignedBy{}{Суровцев М.А.}

\titleAgreedBy{
	\parbox[t]{7cm} {
		\centerline{Академический руководитель ОП}
		\centerline{Системная и программная инженерия}
		\centerline{профессор ДПИ}
}}{Д.В. Александров}
\titleApprovedBy{
	\parbox[t]{7cm} {
		\centerline{Академический руководитель}
		\centerline{образовательной программы}
		\centerline{«Программная Инженерия»}
		\centerline{профессор, канд. техн. наук}
}}{В. В. Шилов}
\titleName{Клиент-Серверное Приложение для Управления Скидками в Розничных Сетях}
\workTypeId{RU.*** TODO ***.506900 T3 01-1}

\titleSubname{Техническое задание}

\begin{document}
	\pagenumbering{arabic}
	
	\section{Аннотация}
	
\tab[0.75cm] Техническое задание – это основной документ, оговаривающий набор требований и
порядок создания программного продукта, в соответствии с которым производится разработка
программы, ее тестирование и приемка.

Настоящее Техническое задание на разработку ``Клиент-Серверное Android-Приложение для Управления Скидками в Розничных Сетях'' содержит следующие разделы: ``Введение'', ``Основания для разработки'',
``Назначение разработки'', ``Требования к программе'', ``Требования к программным документам'',
``Технико-экономические показатели'', ``Стадии и этапы разработки'', ``Порядок контроля и
приемки'' и приложения.

В разделе ``Введение'' указано наименование и краткая характеристика области применения
``Клиент-Серверного Android-Приложения для Управления Скидками в Розничных Сетях''.

В разделе ``Основания для разработки'' указан документ на основании, которого ведется
разработка и наименование темы разработки.

В разделе ``Назначение разработки'' указано функциональное и эксплуатационное
назначение программного продукта.

Раздел ``Требования к программе'' содержит основные требования к функциональным
характеристикам, к надежности, к условиям эксплуатации, к составу и параметрам технических
средств, к информационной и программной совместимости, к маркировке и упаковке, к
транспортировке и хранению, а также специальные требования.

Раздел ``Требования к программным документам'' содержит предварительный состав
программной документации и специальные требования к ней.

Раздел ``Технико-экономические показатели'' содержит ориентировочную экономическую
эффективность, предполагаемую годовую потребность, экономические преимущества разработки
``Клиент-Серверного Android-Приложения для Управления Скидками в Розничных Сетях''.

Раздел ``Стадии и этапы разработки'' содержит стадии разработки, этапы и содержание
работ.

В разделе ``Порядок контроля и приемки'' указаны общие требования к приемке работы.

Настоящий документ разработан в соответствии с требованиями:\\
1) ГОСТ 19.101-77 Виды программ и программных документов \cite{gost_types_of_software};\\
2) ГОСТ 19.102-77 Стадии разработки \cite{gost_stages_of_devel};\\
3) ГОСТ 19.103-77 Обозначения программ и программных документов \cite{gost_marking_software};\\
4) ГОСТ 19.104-78 Основные надписи \cite{gost_main_signs};\\
5) ГОСТ 19.105-78 Требования к программным документам \cite{gost_demands_for_docs};\\
6) ГОСТ 19.201-78 Техническое задание. Требования к содержанию и оформлению \cite{gost_tz}.\\

Изменения к данному Техническому заданию оформляются согласно ГОСТ 19.603-78 \cite{gost_main_rules_change},
Перед прочтением данного документа рекомендуется ознакомиться с терминологией,
приведенной в Приложении 1 настоящего технического задания.

	
	\newpage
	\tableofcontents
	
	% --- add my custom headers ---
	
	\newpage
	\section{Введение}
	\subsection{Наименование программы}
Наименование программы на русском:
``Клиент-Серверное Android-Приложение для Управления Скидками в Розничных
Сетях''. \\
Наименование на английском:
``The Client-Server Android Application for Managing the Products' Discounts in
Retail Networks''.


\subsection{Краткая характеристика}
Программа предназначена для пользователей смартфонов на базе платформы Android.
Цель работы - создать удобное приложения для составления списков покупок,
добавления в них любых товаров (даже тех, которых нет в магазине), отслеживания
акций, а так же просмотра всех акций в нескольких магазинах. Это позволит
пользователям экономить свои средства на ежедневных покупках и быть в курсе
актуальных акций магазинов.  Серверная часть приложения должна представлять из
себя web-crawler для сбора данных с сайтов розничных сетей. Crawler должен
иметь модульную структуру, для быстрого восстановления после изменения дизайна
сайтов. Должен быть реализован механизм взаимодействия с клиентами посредством
обменивания файлами в формате JSON.\\
Главной чертой данного приложения
является его лёгкая, быстрая масштабируемость и модульность программного кода.



	\newpage
	\section{Основания для разработки}
	\subsection{Документ, на основании которого ведется разработка}
Приказ декана ФКН И.В. Аржанцева
\textnumero ХХХХХХХ от ХХ.ХХ.2017 ``ХХХХХХХХХ''


\subsection{Наименование темы разработки}
Наименование темы: ``Клиент-серверное-приложение для управления скидками в розничных сетях''. \\
Наименование темы на английском: ``The Client-Server Application for Managing the Products' Discounts in Retail Networks''. \\

	
	\newpage
	\section{Назначение разработки}
	\subsection{Функциональное назначение}
Продукт является приложением ``на стыке онлайна и оффлайна'', позволяющим планировать
покупки в магазине с учетом актуальных скидок. 
К функциональным возможностям программы относятся:
просмотр списка актуальных акций,
составление и редактирование списка покупок, 
поиск ближайшего магазина на карте,
регистрация пользователей,
возможность отправки списка покупок между пользователями.

\subsection{Эскплутационное назначение}
Программа предназначена для запуска на персональных компьютерах (веб-версия) и 
мобильных устройствах (веб-версия). Программа является 
приложением, позволяющим пользователю искать и подбирать интересующие его товары 
по сниженной цене, и, таким образом, экономить средства. 


% EOF


	
	\newpage
	\section{Требования к программе}
	Требования к программе представлены в документах:
\begin{my_enumerate}
    \item Клиент-Серверное Android-Приложение для Управления Скидками в Розничных Сетях. Техническое Задание.
    \item Клиент-Серверное Web-Приложение для Управления Скидками в Розничных Сетях. Техническое Задание.
\end{my_enumerate}

% EOF


	
	\newpage
	\section{Требования к программной документации}
	\subsection{Предварительный состав программной документации}
\begin{my_enumerate}
    \item «Игра - Эскейп Квест с Использованием Очков Виртуальной Реальности». Техническое задание
    \item «Игра - Эскейп Квест с Использованием Очков Виртуальной Реальности». Пояснительная записка
    \item «Игра - Эскейп Квест с Использованием Очков Виртуальной Реальности». Руководство оператора
    \item «Игра - Эскейп Квест с Использованием Очков Виртуальной Реальности». Программа и методика испытаний
    \item «Игра - Эскейп Квест с Использованием Очков Виртуальной Реальности».  Текст программы
\end{my_enumerate}



	\newpage
	\section{Технико-экономические показатели}
    \subsection{Предполагаемая потребность}
«Игра — Эскейп Квест с Использованием Очков Виртуальной Реальности» может быть использована в игровой сфере. Программу могут использовать люди, нуждающиеся как в отдыхе в мире виртуальной реальности, так и желающие интеллектуально развиться в игровой форме.
\subsection{Экономические преимущества разработки}
На момент принятия решения о написании данного продукта во Всемирной сети Интерент был лишь 1 аналог: игра "Lost in the Kismet".
Преимущества данной разработки перед конкурентом:
\begin{my_enumerate}
\item Совместимость со всеми девайсами Cardboard
\item VR Mode и Normal Mode
\item Оригинальный сюжет
\end{my_enumerate}
	
	\newpage
	\section{Стадии и этапы разработки}
	\subsection{Необходимые стадии разработки}
\subsubsection{Техническое задание}
Этапы разработки:
\begin{my_enumerate}
    \item Обоснование необходимости разработки программы
        \begin{my_enumerate}
            \item постановка задачи
            \item сбор исходных материалов
            \item обоснование необходимости проведения научно-исследовательских работ
        \end{my_enumerate}
    \item Научно-исследовательские работы
        \begin{my_enumerate}
            \item определение структуры входных и выходных данных;
            \item предварительный выбор методов решения задач;
            \item определение требований к техническим средствам.
        \end{my_enumerate}
    \item Разработка и утверждение технического задания
        \begin{my_enumerate}
            \item определение требований к программе
            \item определение стадий, этапов и сроков разработки программы и документациик ней
        \end{my_enumerate}
\end{my_enumerate}


\subsubsection{Технический проект}
Этапы разработки:
\begin{my_enumerate}
    \item Разработка технического проекта
    \begin{my_enumerate}
        \item разработка технического проекта
        \item разработка структуры программы
    \end{my_enumerate}
    \item Утверждение технического проекта
    \begin{my_enumerate}
        \item разработка плана мероприятий по разработке программы
        \item разработка пояснительной записки
    \end{my_enumerate}
\end{my_enumerate}


\subsubsection{Рабочий проект}
\begin{my_enumerate}
    \item Разработка программы
        \begin{my_enumerate}
            \item программирование и отладка программы
            \item создание пакета инсталляции программы
        \end{my_enumerate}
    \item Разработка программной документации
        \begin{my_enumerate}
            \item разработка программных документов в соответствии с требованиями ГОСТ 19.101-77.
        \end{my_enumerate}
     \item Испытания программы
           \begin{my_enumerate}
             \item разработка, согласование и утверждение программы и методики испытаний
            \item корректировка программы и программной документации по результатами испытаний
         \end{my_enumerate}
\end{my_enumerate}

\subsection{Сроки работ и исполнители}

Приложение должно быть разработано к 1 апреля 2018 года, студентом группы БПИ151 
Суровцевым Максимом.


% EOF

	
	\newpage
	\section{Порядок контроля и приемки}
	Контроль и приемка разработки осуществляются в соответствии с документом: 
``Клиент-Серверное Web Приложение для Управления Скидками в Розничных Сетях. 
Программа и методика испытаний''. \\
Испытания проводятся поэтапно, друг за другом, в следующем порядке:
\begin{my_enumerate}
	\item Испытание выполнения требований к программной документации
	\item Испытание выполнения требований к графическому интерфейсу и оформлению программы
	\item Испытание выполнения требований к функциональным характеристикам программы, надежности и корректности ее работы
	\item Испытание выполнения требований к временным характеристикам
\end{my_enumerate}

	
	\newpage
	\section{Приложение 1. Терминология}
	\subsection{Терминология}
\begin{description}
  \item[REST (от англ. Representational State Transfer)] -- способ сетевого
    взаимодействия.  REST API определяет набор функций, к которым разработчики
    могут совершать запросы и получать ответы. Взаимодействие происходит по
    протоколу HTTP.  Преимуществом такого подхода является широкое
    распространение протокола HTTP, поэтому REST API можно использовать
    практически из любого языка программирования.
\end{description}


	
	\newpage
	\section{Приложение 2. Список используемой литературы}
	\subsection{Список используемой литературы}
\begin{my_enumerate}
    \item
    ГОСТ 19.102-77 Стадии разработки. //Единая система программной документации. -М.: ИПК Издательство стандартов, 2001.
    
    \item
    ГОСТ 19.201-78 Техническое задание. Требования к содержанию и оформлению // Единая система программной документации. -М.:ИПК Издательство стандартов, 2001.
    
    \item  ГОСТ 19.404-79 Пояснительная записка. Требования к содержанию и оформлению. //Единая система программной документации. – М.: ИПК Издательство стандартов, 2001
    
    \item
    ГОСТ 19.101-77 Виды программ и программных документов
    //Единая система программной документации. -М.: ИПК Издательство стандартов, 2.: 001.
    
    \item
    ГОСТ 19.103-77 Обозначения программ и программных документов. //Единая система программной документации. -М.: ИПК Издательство стандартов, 2001.
    
    \item
    ГОСТ 19.104-78 Основные надписи //Единая система программной документации. -М.: ИПК Издательство стандартов, 2001.
    
    \item 
    ГОСТ 19.106-78 Требования к программным документам, выполненным печатным способом. //Единая
    система программной документации. – М.: ИПК Издательство стандартов, 2001
    
    \item 
    ГОСТ 19.603-78 Общие правила внесения изменений. //Единая система программной документации. –
    М.: ИПК Издательство стандартов, 2001
    
    \item
    Oculus Documentation [Электронный ресурс]: Режим доступа: https:// developer3.oculus. com/documentation/
    
    \item
    Oculus Developers Blog [Электронный ресурс]: chrispruett – Squeezing Performance out of your Unity Gear VR Game, 2015 - Режим доступа: https://developer3.oculus.com/blog /squeezing-performance-out-of-your-unity-gear-vr-game/
    
    \item 
    Uninty Scripting Reference [Электронный ресурс]: Режим доступа: https: //docs.unity3d. com/ScriptReference/
    
\end{my_enumerate}


	
	% Index
	\newpage
	\eskdListOfChanges

\end{document}


% EOF
