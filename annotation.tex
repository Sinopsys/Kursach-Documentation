
\tab[0.75cm] Техническое задание – это основной документ, оговаривающий набор требований и
порядок создания программного продукта, в соответствии с которым производится разработка
программы, ее тестирование и приемка.

Настоящее Техническое задание на разработку «Игры - Эскейп Квеста с Использованием Очков Виртуальной Реальности» содержит следующие разделы: «Введение», «Основания для разработки»,
«Назначение разработки», «Требования к программе», «Требования к программным документам»,
«Технико-экономические показатели», «Стадии и этапы разработки», «Порядок контроля и
приемки» и приложения.

В разделе «Введение» указано наименование и краткая характеристика области применения
«Игры - Эскейп Квеста с Использованием Очков Виртуальной Реальности».

В разделе «Основания для разработки» указан документ на основании, которого ведется
разработка и наименование темы разработки.

В разделе «Назначение разработки» указано функциональное и эксплуатационное
назначение программного продукта.

Раздел «Требования к программе» содержит основные требования к функциональным
характеристикам, к надежности, к условиям эксплуатации, к составу и параметрам технических
средств, к информационной и программной совместимости, к маркировке и упаковке, к
транспортировке и хранению, а также специальные требования.

Раздел «Требования к программным документам» содержит предварительный состав
программной документации и специальные требования к ней.

Раздел «Технико-экономические показатели» содержит ориентировочную экономическую
эффективность, предполагаемую годовую потребность, экономические преимущества разработки
«Игры - Эскейп Квеста с Использованием Очков Виртуальной Реальности».

Раздел «Стадии и этапы разработки» содержит стадии разработки, этапы и содержание
работ.

В разделе «Порядок контроля и приемки» указаны общие требования к приемке работы.

Настоящий документ разработан в соответствии с требованиями:
1) ГОСТ 19.101-77 Виды программ и программных документов [1];
2) ГОСТ 19.102-77 Стадии разработки [2];
3) ГОСТ 19.103-77 Обозначения программ и программных документов [3];
4) ГОСТ 19.104-78 Основные надписи [4];
5) ГОСТ 19.105-78 Общие требования к программным документам [5];
6) ГОСТ 19.106-78 Требования к программным документам, выполненным печатным способом
[6];
7) ГОСТ 19.201-78 Техническое задание. Требования к содержанию и оформлению [7].
Изменения к данному Техническому заданию оформляются согласно ГОСТ 19.603-78 [8],
ГОСТ 19.604-78 [9].
Перед прочтением данного документа рекомендуется ознакомиться с терминологией,
приведенной в Приложении 1 настоящего технического задания.